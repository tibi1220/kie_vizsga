\documentclass[main.tex]{subfiles}

\begin{document}
  % ----------------------------------------------------------------------------
  % ------------------------------ Section 09 ----------------------------------
  % ----------------------------------------------------------------------------
  \section{Tétel – Az adatbázisok alapjai} %9
    
  \fogalom{Adatbázis}
  \begin{itemize}
    \item \zkod{adatok} – nyers tények, feldolgozatlan információ
    
    \item \zkod{információ} – feldolgozott adat
    \begin{itemize}
      \item információs rendszerek hozzák létre,
      keresik vissza, dolgozzák fel.
    \end{itemize}

    \item az adatok és információk közötti különbség
    nem strukturális, hanem \zkod{funkcionális}
    
    \item az \zkod{adatbázis} adatok gyűjteménye,
    amelyet egy adatbázis-kezelő rendszer kezel, tehát
    nemcsak az adatok rendezett tátolását, hanem azok
    kezelését is lehetővé teszi, mert kapcsolatok nélkül
    az adatok eltérően értelmezhetőek

    \item az eltárolt adatok struktúrája a
    kezelő rendszer együttese

    \item az adatok között meghatározott kapcsolatok vannak
    
    \item alapfunkciók: \zkod{létrehozás}, adatok \zkod{mentés}e,
    \zkod{lekérdezések}, \zkod{adatvédelem}

    \item a \zkod{DBMS} segítségével lehetséges a tárolt adatok
    \zkod{definiálás}a, \zkod{kezelés}e,
    \zkod{karbantartás}a, \zkod{felügyelet}e
  \end{itemize}

  \fogalom{Követelmények}
  \begin{itemize}
    \item \zkod{DDL} – új adatok létrehozása adatdefiníciós
    nyelv segítségével
    
    \item \zkod{SQL} – meglévő adatok lekérdezése, módosítása
    lekérdező vagy adatmanipulációs nyelvvel

    \item támogassa az adatok \zkod{hosszú távú},
    \zkod{biztonságos} tárolását

    \item felügyelje a több felhasználó által egy
    időben történő hozzáférését

    \item \zkod{adatintegritás} \tabto{4.5cm} – \tabto{5.5cm}
    érvényesség, helyesség, ellentmondás-mentesség
    
    \item \zkod{rugalmasság} \tabto{4.5cm} – \tabto{5.5cm}
    adatok egyszerűen módosíthatóak
    
    \item \zkod{hatékonyság} \tabto{4.5cm} – \tabto{5.5cm}
    gyors és hatékony keresés, módosíthatóság
    
    \item \zkod{adatfüggetlenség} \tabto{4.5cm} – \tabto{5.5cm}
    hardver és szoftverfüggetlenség
    
    \item \zkod{adatbiztonság} \tabto{4.5cm} – \tabto{5.5cm}
    védelem a hardver és szoftverhibáb ellen
    
    \item \zkod{adatvédelem} \tabto{4.5cm} – \tabto{5.5cm}
    illetéktelen felhasználókkal szemben
    
    \item \zkod{osztott hozzáférés} \tabto{4.5cm} – \tabto{5.5cm}
    több felhasználó egyidejű hozzáférése

    \item \zkod{integritási kényszerek} \tabto{5cm} – \tabto{5.5cm}
    szabályok, amiket figyelembe kellm venni

    \item \zkod{tranzakciók} \tabto{4.5cm} – \tabto{5.5cm}
    felhasználó általi változtatás nem végleges azonnal
  \end{itemize}

  \fogalom{Adatmodellezés}
  \begin{itemize}
    \item az adatmodellezés segítséget nyújt a
    környezővilág megértésében és leképezésében,
    a lényeges jellemzők kiemelésében

    \item az adatmodell az \zkod{adatok} és az
    azok \zkod{közötti összefüggés}ek leírására szolgál

    \item a \zkod{modell} olyan mesterséges rendszer,
    amely felépítésében és viselkedésében megegyezik
    a vizsgált létező rendszerrel

    \item adatmodellnek nevezzük az adatok struktúrájának
    (felépítésének) leírására szolgáló modelleket.
  \end{itemize}

  \fogalom{Egyed-kapcsolat modell}
  \begin{itemize}
    \item az \zkod{ETK adatmodell} a lenti három fogalom együttese
    
    \item \zkod{egyed} – az információs rendszert
    felépítő személyek, tárgyak, események 
    \begin{itemize}
      \item nem elszigetelten, hanem kapcsolatban állnak
      egymással és más objetumokkal
    \end{itemize}
    
    \item \zkod{tulajdonság} – egyedeket jellemzik,
    egy érték, aminek tulajdonságtípusa van
    \begin{itemize}
      \item \zkod{azonosító}
      \tabto{6.5cm} – \tabto{7.5cm}
      egyedi, nem ismétlődhet (\nkod{ID})
      
      \item \zkod{leíró tulajdonság}
      \tabto{6.5cm} – \tabto{7.5cm}
      ismétlődhet (\nkod{név})

      \item \zkod{gyengén jellemző tulajdonság}
      \tabto{6.5cm} – \tabto{7.5cm}
      lehet üres is (\nkod{kedvenc szín})

      \item \zkod{kapcsoló tulajdonság}
      \tabto{6.5cm} – \tabto{7.5cm}
      itt leíró, ott azonosító (\nkod{szül hely})
    \end{itemize}

    \item \zkod{kapcsolat} – 2 egyed közötti viszony
    \begin{itemize}
      \item lehet többszintű, bonyolult, melyben
      több rendszer is kapcsolatban áll egymással
    \end{itemize}
  \end{itemize}

  \fogalom{Kapcsolat típusok}
  \begin{itemize}
    \item (\zkod{1-1}) – "\zkod{egy az egyhez}"
    \begin{itemize}
      \item egy egyedtípus egy egyedéhez egy másik
      egyedtípus csak egyetlen egyede kapcsolódhat,
      és fordítva is igaz
      (\nkod{osztály–osztályfőnök})
    \end{itemize}

    \item (\zkod{1-N}) – "\zkod{egy a többhöz}"
    \begin{itemize}
      \item egy egyedtípus egy egyedéhez egy másik
      egyedtípus több egyede is kapcsolódhat,
      de fordítva nem igaz
      (\nkod{osztály–tanuló})
    \end{itemize}

    \item (\zkod{N-M}) – "\zkod{több a többhöz}"
    \begin{itemize}
      \item egy egyedtípus egy egyedéhez egy másik
      egyedtípus több egyede is kapcsolódhat,
      de fordítva is igaz
      (\nkod{osztály–tanár})
    \end{itemize}

    \item az \zkod{adatbázis} fogalma a kapcsolatok alapján:
    \begin{itemize}
      \item véges számú egyedek és azok véges számú tulajdonságainak
      és kapcsolatainak adatmodell szerinti szervezett együttese
    \end{itemize}

    \begin{figure}[H]
      \flushright
      \begin{tikzpicture}[node distance=8em]
        \node[entity] (ent1) {Egyed${}_1$};
        \node[relationship] (rel1) [right of=ent1] {Kapcsolat${}_1$} edge[thick] (ent1);
        \node[attribute] (attr1) [below left of=rel1] {\key{Kulcstul}} edge[<-, thick] (rel1);
        \node[attribute] (attr2) [below of=rel1] {Többértékű} edge[<<-, thick] (rel1);
        \node[attribute] (attr3) [below right of=rel1] {\xcancel{Összetett}} edge[<-,thick] (rel1);
        \node[entity] (ent2) [right of=rel1] {Egyed${}_2$} edge[thick] (rel1);
        \node[relationship] (rel2) [right of=ent2] {Kapcsolat${}_2$} edge[thick] (ent2);
        \node[attribute] (attr4) [below of=rel2] {Egyszerű} edge[<-, thick] (rel2);
        \node[attribute] (attr5) [below right of=rel2] {Egyértékű} edge[<-, thick] (rel2);
      \end{tikzpicture}
    \end{figure} %tikz

    \item a \zkod{kapcsolat foka} lehet\dots
    \begin{itemize}
      \item \zkod{Unáris} \tabto{2cm} – \tabto{2.5cm} rekurzív
      
      \item \zkod{Bináris} \tabto{2cm} – \tabto{2.5cm} két résztvevős

      \item \zkod{Trináris} \tabto{2cm} – \tabto{2.5cm} három résztvevős
    \end{itemize}
  \end{itemize}


  % ----------------------------------------------------------------------------
  % ------------------------------ Section 10 ----------------------------------
  % ----------------------------------------------------------------------------
  \section{Tétel – Relációs adatbázismodell} %10

  \fogalom{Adatbázis szerkezetek}
  A leggyakoribb adatszerkezetek:
  \zkod{hierarchikus},
  \zkod{hálós},
  \zkod{relációs}.

  \vspace{.5em}
  \zkod{Hierarchikus} esetben csak egy a többhöz
  (\zkod{1-N}) kapcsolatok képezhetőek le.
  Az adatok a tárolt hierarchia szerint érhetőek el.
  Fastruktúrával szemléltethető,
  hiszen az adatokat alá-fölérendeltségi
  viszonnyal meghatározható szerkezettel írjuk le.

  \vspace{.5em}
  \zkod{Hálós} adatbázismodell esetén a kapcsolatok
  gráfokkal írhatóak le, ahol a csomópontokat élekkel
  kötjük össze. Csak a tárolt kapcsolat mentén járható be.
  A modellel egy a többhöz (\zkod{1-N}) és több a többhöz 
  (\zkod{N-M}) kapcsolatot és leírhatunk.

  \fogalom{Reláció jellemzői}

  \zkod{Relációs modell} esetén az adatokat \zkod{táblázat}os
  formában tároljuk. Nincsenek előre meghatározott kapcsolatok.
  \begin{itemize}
    \item egyszerűen értelmezhető, átlátható
    
    \item rugalmas, könnyen kezelhető
    
    \item relációk kezelése relációs algebrával
  \end{itemize}

  A \zkod{reláció} egy \zkod{adattábla}. (táblázat)
  \begin{itemize}
    \item \zkod{rekord} \tabto{2.5cm} – \tabto{3.2cm}
    a táblázat sorai
    
    \item \zkod{attribútum} \tabto{2.5cm} – \tabto{3.2cm}
    a táblázat oszlopai

    \item \zkod{mező} \tabto{2.5cm} – \tabto{3.2cm}
    a sorok és oszlopok metszetei
  \end{itemize}

  A reláció rekordjaiban tároljuk a logikailag összefüggő adatokat.
  A relációban tárolt rekordok száma a reláció \zkod{számossága}.
  Az oszlopokban azonos tulajdonságokra vonatkozó adatok jelennek meg.
  Egy tábla nem tartalmazhat azonos nevű oszlopot.
  Az oszlopok száma a reláció \zkod{foka}. Követelmények:
  \begin{itemize}
    \item minden rekordja különböző
    \item nincs két azonos attribútum
    \item minden rekord mezőszerkezete azonos
    \item a rekordok és attribútumok sorrendje tetszőleges
  \end{itemize}

  \fogalom{A relációs algebra műveletei}
  \begin{enumerate}
    \item \zkod{adatkezelő} műveletek
    \begin{itemize}
      \item adatbevitel, törlés, adatmódosítás
    \end{itemize}

    \item \zkod{adatlekérdező} műveletek
    \begin{itemize}
      \item relációs algebra műveleteivel,
      mindig új relációt eredményez
    \end{itemize}
  \end{enumerate}

  A \zkod{relációs algebra} műveletei lehetnek
  egy vagy többoperandusúak. Az előbbit egy, az
  utóbbit pedig több reláción végezzük el.

  \vspace{1em}
  \begin{minipage}[t]{0.5\textwidth}
    \begin{itemize}
      \item \zkod{szelekció} (kiválasztás) \\
      \tabto{5mm} rekordokat választunk ki
      
      \item \zkod{projekció} (vetítés) \\
      \tabto{5mm} attribútumokat választunk ki

      \item \zkod{kiterjesztés} \\
      \tabto{5mm} matematikai műveletekkel új oszlop

      \item \zkod{csoportosítás} \\
      \tabto{5mm} attribútumok alapján csoport, \\
      \tabto{5mm} majd hozzá érték
    \end{itemize}
  \end{minipage}\hfill
  \begin{minipage}[t]{0.5\textwidth}
    \begin{itemize}
      \item \zkod{Descartes-szorzat} \\
      \tabto{5mm} 2 reláció sorainak összes kombinációja

      \item \zkod{összekapcsolás} \\
      \tabto{5mm} összekapcsolás attribútum alapján

      \item \zkod{unió}

      \item \zkod{metszet}

      \item \zkod{különbség} \\
      \tabto{5mm} nem kommutatív
    \end{itemize}
  \end{minipage}

  Az azonosítás történhet \zkod{kulcs} alapján,
  amely egyértelműen azonosítja az egyedet az egyedhalmazon belül.
  Amennyiben egyetlen attribútumból áll, akkor \zkod{egyszerű},
  egyébként \zkod{összetett}.
  Megadható több kulcs is, amire szükségünk van az adott
  feladatnál, azt \zkod{elsődleges} kulcsnak nevezzük,
  a többi mind \zkod{másodlagos}. \zkod{Idegen} kulcs egy
  reláció olyan attribútumai, melyek egy másikban elsődlegesek.

  \fogalom{SQL alapok}
  Hogy elkerüljük az \zkod{anomáliá}kat, az adatbázisokat
  \zkod{normalizálni} szokták. Ennek lényege, hogy a
  változtatási anomáliák megszűnjenek. (módosítási, beírási,
  törlési) A normalizálásnak több szintje létezik.
  Minden \zkod{relációs séma} megköveteli legalább
  az első normálformát. Gyakorlatban a harmadikig szokták.
  \begin{itemize}
    \item \zkod{1. normál forma}
    \begin{itemize}
      \item ha a mezők függéseinek rendszerében
      létezik egy olyan kulcs, amelytől minden
      más mező függ, azaz minden mezője 
      funkcionálisan függ a kulcsmező csoporttól
    \end{itemize}
    
    \item \zkod{2. normál forma}
    \begin{itemize}
      \item nincs benne részleges függés, azaz bármely nem
      kulcs mező a teljes kulcstól függ, de nem függ a
      kulcs bármely részhalmazától
    \end{itemize}

    \item \zkod{3. normál forma}
    \begin{itemize}
      \item nem áll fenn tranzitív függőség, azaz
      a nem kulcs mezők nem függnek egymástól,
      tehát nincs funkcionális függőség a \pkod{nem}
      elsődleges attribútumok között.
    \end{itemize}
  \end{itemize}

  {\large \zkod{Definíció}}:
  \begin{itemize}
    \item \kkod{CREATE TABLE} \fkod{table\textunderscore{}name
    (column1 datatype cond,\dots);}
    \\ \hspace*{\fill}
    \nkod{objektum létrehozása}

    \item \kkod{DROP TABLE} \fkod{table\textunderscore{}name;}
    \\ \hspace*{\fill}
    \nkod{objektum megszűntetése}

    \item \kkod{ALTER TABLE} \fkod{table\textunderscore{}name}
    \kkod{ADD|MODIFY} \fkod{(column1 datatype cond|cond);}
    \\ \hspace*{\fill}
    \nkod{objektum séma módosítása}
  \end{itemize}

  \begin{minipage}[c]{0.5\textwidth}
    \begin{table}[H]
      \centering\begin{tabular}{|c|}
        \hline
        \fkod{PRIMARY KEY} \\
        \fkod{NOT NULL} \\
        \fkod{UNIQUE} \\
        \fkod{CHECK cond} \\
        \fkod{REFERENCING table\textunderscore{}name} \\
        \hline
      \end{tabular}
    \end{table}
  \end{minipage}\hfill
  \begin{minipage}[c]{0.5\textwidth}
    \begin{table}[H]
      \centering\begin{tabular}{|c|}
        \hline
        \fkod{CHAR(n)}
        \\ 
        \fkod{NUMBER(n,m)}
        \\ 
        \fkod{DATE}
        \\ \hline
      \end{tabular}
    \end{table}
  \end{minipage}\hfill

  \vspace{1em}
  {\large \zkod{Módosítás}}:
  \begin{itemize}
    \item \kkod{INSERT INTO} \fkod{table\textunderscore{}name}
    \kkod{VALUES} \fkod{(field=value);}
    \\ \hspace*{\fill}
    \nkod{rekord felvitele}

    \item \kkod{DELETE FROM} \fkod{table\textunderscore{}name}
    \kkod{WHERE} \fkod{cond;}
    \\ \hspace*{\fill}
    \nkod{rekord tőrlése}

    \item \kkod{UPDATE} \fkod{table\textunderscore{}name}
    \kkod{SET} \fkod{field=value,\dots} \kkod{WHERE} \fkod{cond;}
    \\ \hspace*{\fill}
    \nkod{rekord módosítása}
  \end{itemize}

  {\large \zkod{Adatok lekérdezése}}:
  \begin{itemize}
    \item \kkod{SELECT} \fkod{culomn1, column2, \dots}
    \kkod{FROM} \fkod{table\textunderscore{}name1, table\textunderscore{}name,}
    \kkod{WHERE} \fkod{cond;}

    \vspace{.5em}
    \begin{table}[H]
      \centering\begin{tabular}{|l r|}
        \hline
        \kkod{GROUP BY} & csoportosítás \\
        \kkod{HAVING} \fkod{cond} & \hspace{2em} megszorítás \\
        \kkod{ORDER BY} & rendezés \\
        \hline
      \end{tabular}
    \end{table}

    \item \kkod{SELECT} \fkod{column} \kkod{FROM}
    \fkod{table\textunderscore{}name;}
    \hfill \nkod{projekció}

    \item \kkod{SELECT} \fkod{column} \kkod{FROM}
    \fkod{table\textunderscore{}name}
    \kkod{WHERE} \fkod{cond;}
    \hfill \nkod{szelekció}
    
    \item \kkod{SELECT} \fkod{*} \kkod{FROM}
    \fkod{table\textunderscore{}name1, table\textunderscore{}name2;}
    \hfill \nkod{Descartes-szorzat}
    
    \item \kkod{SELECT} \fkod{expr culomn,\dots}
    \kkod{FROM} \fkod{table\textunderscore{}name;}
    \hfill \nkod{kiterjesztés}

    \item \kkod{SELECT} \fkod{aggregation}
    \kkod{FROM} \fkod{table\textunderscore{}name;}
    \hfill \nkod{aggregáció megadása}

    \vspace{.5em}
    \begin{table}[H]
      \centering\begin{tabular}{|c|}
        \hline
        \kkod{SUM}\fkod{(expr)} \\
        \kkod{COUNT}\fkod{(expr)} \\
        \kkod{MIN}\fkod{(expr)} \\
        \kkod{AVG}\fkod{(expr)} \\
        \kkod{MAX}\fkod{(expr)} \\
        \hline
      \end{tabular}
    \end{table}

    \item \kkod{SELECT} \fkod{aggregation}
    \kkod{FROM} \fkod{table\textunderscore{}name;}
    \kkod{GROUP BY} \fkod{expr;}
    \\ \hspace*{\fill}
    \nkod{aggregáció, csoportképzés}

    \item \kkod{SELECT} \fkod{column}
    \kkod{FROM} \fkod{table\textunderscore{}name;}
    \kkod{ORDER BY} \fkod{column1 mode1, \dots;}
    \\ \hspace*{\fill}
    \nkod{eredmény rekordok rendezése}

    \begin{table}[H]
      \centering\begin{tabular}{|c|}
        \hline
        \kkod{ASC} \\
        \kkod{DESC}\\
        \hline
      \end{tabular}
    \end{table}
  \end{itemize}

  {\large \zkod{Adatok lekérdezése}}:

  \begin{minipage}[c]{0.5\textwidth}
    \begin{table}[H]
      \centering\begin{tabular}{|l r|}
        \hline
        \kkod{=} & egyenlő \\
        \kkod{\textless\textgreater,\string^=} & nem egyenlő \\
        \kkod{\textgreater} & nagyobb \\
        \kkod{\textgreater=} & nagyobb egyenlő \\
        \kkod{\textless} & kisebb \\
        \kkod{\textless=} & kisebb egyenlő \\
        \hline
      \end{tabular}
    \end{table}
    \begin{table}[H]
      \centering\begin{tabular}{|l r|}
        \hline
        \kkod{LIKE} \zkod{'a\%'}
        & \zkod{'a'}-val kezdődik \\
        \kkod{LIKE} \zkod{'x\textunderscore'}
        & \zkod{'x'}-val kezdődik, 2 betű \\
        \kkod{LIKE} \zkod{'\%a\%'}
        & \zkod{'a'}-t tartalmaz \\
        \kkod{LIKE} \zkod{'\textunderscore{}a\%x'}
        & 2. betű \zkod{'a'}, \zkod{'x'}-re végződik \\
        \hline
      \end{tabular}
    \end{table}
  \end{minipage}\hfill
  \begin{minipage}[c]{0.5\textwidth}
    \begin{table}[H]
      \centering\begin{tabular}{|l r|}
        \hline
        \kkod{BETWEEN}\fkod{ x }\kkod{AND}\fkod{ y}
        & \hspace{.25em} adott értékek közé esik \\
        \kkod{IN}\fkod{(a, b, c,\dots)} & értékek között van \\
        \kkod{LIKE} \fkod{sample} & hasonlít a mintára \\
        \hline
      \end{tabular}
    \end{table}
    \begin{table}[H]
      \centering\begin{tabular}{|l r|}
        \hline
        \kkod{NOT} & tagadás \\
        \kkod{AND} & és \\
        \kkod{OR} & vagy \\
        \hline
      \end{tabular}
    \end{table}
  \end{minipage}\hfill

\end{document}