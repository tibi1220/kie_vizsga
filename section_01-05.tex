\documentclass[main.tex]{subfiles}

\begin{document}
  % ----------------------------------------------------------------------------
  % ------------------------------ Section 01 ----------------------------------
  % ----------------------------------------------------------------------------
  \section{Tétel – Az objektum-orientált programozás alapjai} %1

  \fogalom{Objektum}
  Az objektum (\kkod{object}) egy rendszer egyedileg
  azonosítható szereplője adatokkal és működéssel.
  

  \fogalom{Osztály}
  Az osztály (\kkod{class}) megegyező szerkezetű,
  hasonló viselkedésű elemek mintája.
  Ez egy felhasználó által definiált típus.
  Az osztályhoz tartozó példány (\kkod{instance}) az objektum.
  Adatszerkezetből és tagfüggvényekből áll.
  Alaphelyzetben az adattagok és a tagfüggvények
  nem láthatóak.


  \fogalom{Struktúra}
  A struktúrák (\kkod{struct}) adattagjai kívülről
  alaphelyzetben elérhetőek, ellentétben az osztályokkal.


  \fogalom{Tagfüggvények}
  Az adattípussal kapcsolatban álló (az adattagokon működő)
  függvények lehetnek belül, vagy kívül definiáltak.
  Lehetnek továbbá \kkod{inline} függvények.
  (Hívás helyén lesznek lefordítva.)


  \fogalom{Konstruktor}
  Függvény az objektumok előkészítésére, esetleg inicializálása.
  Ugyanaz a neve mint az osztálynak, nincs típusa,
  sem visszatérési értéke. Létrehozás után nem hívható.
  Ha nem adjuk meg akkor is létezik alapértelmezett. (\kkod{default})
  Másoló konstruktor megadása konstans referenciával lehetséges.


  \fogalom{Destruktor}
  Az objektumok megszűnésekor automatikusan meghívódó függvény.
  A neve: \kkod{\textasciitilde ClassName}. Nincsenek paraméterei.
  Nem terhelhető túl. Statikus \kkod{instance} esetén a
  \kkod{scope}-on kívül, dinamikus esetben a \kkod{delete}
  kulcsszó esetén hívódik meg.


  \fogalom{Statikus tagok}
  A statikus (\kkod{static}) tagok nem az objektumhoz,
  hanem az osztályhoz vannak kötve. Az osztályban nem definiálható,
  névtér (\kkod{namespace}) szintjén kell ezt megtenni.


  \fogalom{Friend függvények}
  A \kkod{friend} függvények segítségével az osztály rejtett tagjai
  is elérhetőek.


  \pagebreak
  % ----------------------------------------------------------------------------
  % ------------------------------ Section 02 ----------------------------------
  % ----------------------------------------------------------------------------
  \section{Tétel – Az objektum-orientált programozás alapjai} %2

  \fogalom{Operátorok túlterhelése}
  Szinte minden operátor túlterhelhető, kivéve:
  \kkod{::}, \kkod{.}, \kkod{.*}, \kkod{?},
  \kkod{sizeof()}, \kkod{typeid()}.
  Új műveleti jel nem hozhetó létre.
  Egyoperandusú művelet esetén:
  \begin{itemize}
    \item \kkod{opetaror++();}
    \hspace{11.5mm} $\rightarrow$ \hspace{5mm}
    \kkod{myVar++} \hspace{5mm} (Utótag)

    \item \kkod{opetaror++(int);}
    \hspace{5mm} $\rightarrow$ \hspace{5mm}
    \kkod{++myVar} \hspace{5mm} (Előtag)
  \end{itemize}


  \fogalom{C++ IO}
  Az \kkod{std namespace}-t használjuk, az alábbi fejlécekkel:
  \kkod{iostream}, \kkod{iomanip}.
  \begin{lstlisting}
    #include <iostream>
    #include <iomanip>

    cin.get();
    cout.put('\n');

    cout << "Hello World"
    string tmp;
    cin >> tmp;

    cout.flags(ios_base::hex)
      // (no)boolalpha
      // left, right
      // dec, hex, oct
      // fixed, scientific
    cout.precision(2)

    cout << setw(5) << setprecision(2) << 12.345;

    // Overloading
    friend ostream& operator<< (ostream& os, const type& myVar);
    friend istream& operator>> (istream& is, type& myVar);
  \end{lstlisting}


  \fogalom{New, delete}
  A \kkod{new} és \kkod{delete} kulcsszavak segítségével
  dinamikus példányokat hozhatunk létre, és törölhetünk.
  Ezek is túlterhelhetőek:
  \begin{lstlisting}
    void* operator new(size_t size){
      // { ... }
      return new typename[size];
    }

    void operator delete (void* p, size_t size){
      // { ... }
      ::delete p;
    }
  \end{lstlisting}


  \fogalom{Osztály hierarchiák}
  Az újrahasznosítható szoftver alapja.
  Többszörös öröklődés (\kkod{inheritance}) lehetséges.


  % ----------------------------------------------------------------------------
  % ------------------------------ Section 03 ----------------------------------
  % ----------------------------------------------------------------------------
  \section{Tétel – Az objektum-orientált programozás alapjai} %3

  \fogalom{Öröklődés}
  \begin{table}[htbp]
    \centering\begin{tabular}{| c || c | c | c |}
      \hline
       & \kkod{public} & \nkod{protected} & \pkod{private} \\
       \hline \hline
       \kkod{public} & \kkod{public} & \nkod{protected} & \pkod{-}\\
       \hline
       \nkod{protected} & \nkod{protected} & \nkod{protected} & \pkod{-}\\
       \hline
       \pkod{private} & \pkod{private} & \pkod{private} & \pkod{-}\\
       \hline
      \end{tabular}
  \end{table}

  \begin{itemize}
    \item osztály más osztályok tulajdonságait/viselkedését
    is magába integrálja

    \item módosított viselkedésű osztály az eredeti kód másolata,
    hivatkozása nélkül

    \item minden változás automatikusan végigmegy a hierarchián
    
    \item az ős neve a bázis osztály (\kkod{base/parent class})
    
    \item az utód neve leszármazott osztály (\kkod{derived/child class})
    
    \item konstruktor, destruktor, barát függvények, illetve
    az \kkod{operator=} túlterhelése nem öröklődnek,
    öröklődnek viszont az adattagok, tagfüggvények és
    a többi túlterhelt operátor
  \end{itemize}


  \fogalom{Egységbe zárás}
  Az objektum egységbe zárja (\kkod{encapsulation})
  az adatokat és a programokat. Magába foglalja a
  külvilág felé mutatott viselkedést.  A belső struktúrája,
  állapota, adatai és kezelőfüggvényei kívülről nem láthatóak.
  (\kkod{data hiding})


  \fogalom{Protected osztálytagok}
  A \kkod{protected} tagváltozók öröklődés esetén
  nyilvánosak az utódosztály számára,
  de kívülről nem elérhetőek.


  \fogalom{Kompozíció, aggregáció}
  Kompozíció esetén egy meglévő osztályt
  tagobjektumként használunk
  fel egy másik osztályban. (statikus példány)
  Aggregáció esetén pointert vagy referenciát
  használunk. (dinamikus példány)
  A különbség akkor lép fek, ha a felhasznált
  osztályt módosítani kezdjük.


  \fogalom{Többszörös öröklődés}
  Többszörös öröklődés esetén több ős van.


  \fogalom{Barátság}
  Az alaposztály barátja az utódban az öröklött tagokat éri el.
  Az utód barátja az ős \kkod{public} és \kkod{protected} tagjait éri el.
  \begin{lstlisting}
    friend type myFunc();           // Kulso fuggveny
    friend type MyClass::myFunc();  // Masik osztaly publikus tagfuggvenye
    friend MyClass;                 // Masik osztaly osszes fuggvenye
  \end{lstlisting}



  % ----------------------------------------------------------------------------
  % ------------------------------ Section 04 ----------------------------------
  % ----------------------------------------------------------------------------
  \section{Tétel – Az objektum-orientált programozás alapjai} %4

  \fogalom{Polimorfizmus}
  A szó jelentése: ugyanaz a metódus más-
  és másképpen működik a családfa szintjein.
  A tagfüggvények újradefiniálhatóak.
  Lehet a fordításkor korai, statikus kötés,
  vagy futás közben késői, vagy dinamikus kötés.
  Ősosztály-típusú mutatóval meghívható az ős metódusa a leszármazottra.
  \begin{lstlisting}
    class Parent{};
    class Child : public Parent{
      public:
        Child() : Parent() {};
    };
  \end{lstlisting}

  \fogalom{Virtuális alaposztályok.}
  A \kkod{virtual} kulcsszót használva a tagfüggvények
  átdefiniálhatóak lesznek a leszármazottakban.
  "\kkod{=0}" esetén tisztán virtuális lesz.
  \begin{lstlisting}
    virtual void myFunc = 0;
  \end{lstlisting}
  A virtuális függvények virtuálisak maradnak,
  a származtatott osztályok hierachiájában újra definiálhatjuk
  őket a \kkod{virtul} kulcsszó nélkül. (\kkod{C++11} óta)
  Újradefiniáláskor pontosan egyező függvényt kell megadni
  (név, típus, paraméterlista), egyébként túlterhelés lesz.
  A futás közben dől el, hogy melyik tagfüggvény aktivizálódik.
  Az aktiválás az ős pointerén vagy referenciáján keresztül
  egy a virtuális függvények címeit tartalmazó
  táblázat (\kkod{VMT}) segítségével.

  \fogalom{Abstract osztály}
  Tisztán virtuális függvényt tartalmazó osztály nem példányosítható.
  Ezeket \kkod{abstract} osztályoknak hívjuk.
  Az \kkod{interfész} olyan absztrakt osztály,
  amely csak tisztán virtuális függvényeket tartalmaz.

  \fogalom{Általánosított osztályok}
  Általánosított osztályokat \kkod{template}
  segítségével készíthetünk.
  \begin{lstlisting}
    template<typename T>
    class ClassName{
      T myFunc (T myVar){
        // { ... }
        return myVar
      }
    };
  \end{lstlisting}
  A \kkod{template} minden használó forrás állományban kell.


  \pagebreak
  % ----------------------------------------------------------------------------
  % ------------------------------ Section 05 ----------------------------------
  % ----------------------------------------------------------------------------
  \section{Tétel – Az objektum-orientált programozás alapjai} %5

  \fogalom{Standard Template Library}
  Adatstruktúrákat és algoritmusokat tartalmaz
  \kkod{C++} nyelvhez.

  \fogalom{Tárolók}
  Soros és asszociatív tárolókat tartalmaz az \kkod{STL}.
  A soros tárolók jellemzője,
  hogy megőrzik az elemek beviteli sorrendjét.
  Az asszociatív tárolók elemei egy kulcs alapján érhetők el.
  Lehetnek rendezettek, illetve rendezetlenek.
  Mindegyiknek van \kkod{default} és \kkod{copy}
  konstruktora, \kkod{=} operátora.

  \fogalom{Bejárók}
  A bejárók (\kkod{iterator}) pointer szerű, tároló
  független objektumok. Teljes bejárás esetén a
  \kkod{begin()}/\kkod{rbegin()} és
  \kkod{end()}/\kkod{rend()} használatos.
  Egy pozíciót határoznak meg a tárolóban.
  \kkod{terator}okra alkalmazható függvénysablonok:
  \begin{lstlisting}
    begin(); end();
    advance(<int>count);
    distance(iterator1, iterator2);
    next(iterator, <int>count=1);
    prev(iterator, <int>count=1);
  \end{lstlisting}

  \fogalom{Algoritmusok}
  Algoritmusok a konténerek adatainak kezelésére
  (sorbarakás, keresés, szélsőértékek \dots)

  \fogalom{Függvényobjektumok}
  Függvényobjektumok megvalósítják a függvényhívás
  \kkod{()} operátorát az algoritmusokban.
  Hívásakor az osztály objektumát adjuk át.
  \begin{lstlisting}
    class Even{
      public: bool operator() (int i) const {
        return (x%2) == 0;
      }
    };
    vector<int> numbers{2, 3, 6, 9};
    vector<int>::iterator i = find_if(
      begin(numbers), end(numbers), Even()
    );
  \end{lstlisting}
  \kkod{C++11} óta általánosított függvényobjektumok
  is elérhetőek. Szintén ettől a szabványtól használatosak
  a névtelen (\kkod{lambda}) függvények.
  \begin{itemize}
    \item \kkod{[]\{\};}        \hspace{33pt} – \hspace{8pt}
    nem használ a hívó hatókörében változókat

    \item \kkod{[=]\{\};}       \hspace{27pt} – \hspace{8pt}
    a hívó hatókörében definiált változók másolatát látja
    
    \item \kkod{[\&]\{\};}      \hspace{27pt} – \hspace{8pt}
    a hívó hatókörében definiált változók referenciát látja
    
    \item \kkod{[\&,i]\{\};}  \hspace{15.2pt} – \hspace{8pt}
    a hívó hatókörében definiált változók referenciát,
    \kkod{i} másolatát látja
    
    \item \kkod{[=,\&{}i]\{\};}  \hspace{9pt} – \hspace{8pt}
    a hívó hatókörében definiált változók másolatát,
    \kkod{i} referenciát látja
  \end{itemize}

  \fogalom{Soros tárolók}
  
  Az \kkod{array} adott méretű és típusú tárolótömb
  folytonos memóriaterületen. \kkod{for} ciklussal bejárható.
  \begin{lstlisting}
    // Array
    #include <array>

    array<type, size> myArr { {myVar1, myVar2, myVar3} };

    myArr[0] = 12; // feltoltes [] operatorral, bejarhato for ciklussal

    front(); back(), data(); at(); []         // elem eleres
    begin(); end(); rbegin(); rend();         // iterator
    empty(); size(); max_size()               // meretek
    swap(); fill(); reverse(); accumulate();  // muveletek
  \end{lstlisting}

  Az allokátorobjectumok (\kkod{allocator})a tárolók számára
  lefoglalt memóriaterületet kezelik.
  \begin{lstlisting}
    // Allocator
    #include <memory>

    allocator<type> myAlloc;

    allocate(<size_t> n);
    construct(<pointer> p, <const reference> r);
    deallocate(<pointer> p, <size_t> n);
    destroy();
  \end{lstlisting}

  A \kkod{vector} dinamikus tömbökkel megvalósított soros tároló.
  Folytonos adatterületen pointerrel és bejáróval is bejárható.
  Automatikusan növekszik és csökken a tárolási mérete.
  Az elemek könnyen elérhetők pozíció alapján. (állandó idővel)
  Elemek sorban bejárhatóak (lineáris idővel)
  Elemek illeszthetők/törölhetők a végéről. (konstans idővel)
  Elemek be is illeszthetőek/törölhetőek, de erre más tárolók
  (\kkod{deque}, \kkod{list}) jobb időt produkálnak.
  \begin{lstlisting}
    // Vector
    #include <vector>

    vector<type> myVector1 (count, myVar);
    vector<type> myVector2 (myVector1.begin(), myVector1.end());
    vector<type> myVector3 (myVector1);

    begin(); end(); rbegin(); rend();   // bejarok
    []; at(); front(); back();          // tulajdonsagok
    size(); resize();                   // meretek (ossz)
    capacity(), reverse();              // meretek (hatralevo)
    push_back(), pop_back();            // modositok
    iterator insert();                  // beszuras (nem hatekony)

    // Bejaras
    for(i = myVec.begin(); i != myVec.end(); i++);
    for(int i = 0; i < myVec.size(); i++);
    for(auto i : myVec);
    for(auto& i : myVec);
  \end{lstlisting}

  A \kkod{string} is \kkod{STL} tároló.
  \begin{lstlisting}
    // string
    #include <string>

    at(); []; front(); back(); data();
    begin(); end(); rbegin(); rend();
    empty(); size(), max_size();
    push_back(); copy(); c_str();
    find(); find_first(); insert(); replace();
  \end{lstlisting}

  A \kkod{deque} egy kétvégű sor,
  amely mindkét végén növelhető.
  Konstans idő alatt adhatunk hozzá,
  illetve távolíthatunk el elemet a sor végeiről.
  Nem folytonos adatterületen helyezkedik el.
  (nincs \kkod{capacity()} és \kkod{reverse()})
  Egydimenziós tömböt tartalmazó listában tárolódik.
  Az elemek könnyen elérhetőek pozíció alapján. (állandó idővel)
  (\kkod{[]}, \kkod{at()}, \kkod{front()}, \kkod{back()})
  Elemek sorban bejárhatóak. (lineáris idővel)
  Lassabb, mint a sor. (\kkod{queue})
  
  \vspace{1em}
  \begin{lstlisting}
    // deque
    #include <deque>

    deque<type> myDQ1(count, myVar);
    deque<type> myDQ2{myVar1, myVar2, ...};
    deque<type> myDQ3(myDQ1.begin(), myDQ1.end());
    deque<type> myDQ4(myDQ1);
    
    // Konstans idovel elem hozzaadasa
    push_front(); pop_front();
    push_back(); pop_back();
  \end{lstlisting}
  
  A listák (\kkod{list}) esetén nincsen direkt elérés.
  (\kkod{at()}, \kkod{[]}) Kétirányban láncolt lista.
  Gyors beszúrás törlés, sorbarakás összefésülés.
  \begin{lstlisting}
    // list
    #include <list>

    front(); back(); push_front(); push_back(); pop_front(); pop_back();
    iterator insert();
    sort(); merge(); splice();
  \end{lstlisting}
 
  Az egy irányban láncolt listák (\kkod{forward{}\textunderscore{}list})
  esetén nincsen \kkod{push{}\textunderscore{}back()} és
  \kkod{pop{}\textunderscore{}back()}, viszont van
  \kkod{insert{}\textunderscore{}after()}.
  \begin{lstlisting}
    // forward_list
    #include <forward_list>
  \end{lstlisting}

  \fogalom{Asszociatív tárolók}
  Az asszociatív tárolók (\kkod{map}, \kkod{set})
  absztrakt adattípusok.
  A \kkod{set} egyetlen elem sort tartalmaz, ami a kulcs egyben.
  Nincsen \kkod{[]} operátor.
  \begin{lstlisting}
    // set
    #include <set>

    type array[] = {var1, var2, var3}
    set<type> mySet1(array, array+3)
    set<type> mySet2(mySet1.begin(), mySet1.end())
    set<type> mySet3(mySet1)

    begin(); end(); rbegin(); rend();
    ::iterator; size(); max_size(); insert();
    empty(); erase(); clear(); find();
  \end{lstlisting}

  A \kkod{map} kulcs–érték adatpárokat tartalmaz.
  (a \kkod{pair} sablon szerint) Lehet definiálni az
  összehasonlítást kulcsra és értékre is.
  Van \kkod{[]} operátor. (kulcs kell)
  \begin{lstlisting}
    // map
    #include<map>

    map<type1, type2> myMap1;
    map<type1, type2> myMap2(myMap1.begin(), myMap1.end());
    map<type1, type2> myMap3(myMap1);

    begin(); end(); rbegin(); rend();
    ::iterator; size(); max_size(); insert();
    empty(); erase(); clear(); find();
  \end{lstlisting}

  Léteznek kulcsismétlést megengedő változatok
  (\kkod{multiset}, \kkod{multimap}),
  a keresés ebben az esetben lineáris végrehajtási idejű.
  Mindegyiknek van rendezetlen változata is.
  (\kkod{unordered\textunderscore{}xxx})

  \fogalom{Konténer adapterek}
  A verem (\kkod{stack}) \zkod{FILO} típusú tároló.
  \kkod{Interface} minden standardnak, aminek van
  \kkod{back()}, \kkod{pusk\textunderscore{}back()}
  és \kkod{pop\textunderscore{}back()} művelete.
  (\kkod{vector}, \kkod{deque}, \kkod{list})
  \begin{lstlisting}
    // stack
    #include <stack>

    stack<type, container<type>>;

    empty(); push(); pop(); top(); size()
  \end{lstlisting}

  A sor (\kkod{queue}) \zkod{FIFO} típusú tároló.
  \kkod{Interface} minden standardnak, aminek van
  \kkod{back()}, \kkod{pusk\textunderscore{}back()}
  és \kkod{pop\textunderscore{}back()} művelete.
  (\kkod{vector}, \kkod{deque}, \kkod{list})
  \begin{lstlisting}
    // queue
    #include <queue>

    queue<type, container<type>>;

    empty(); push(); pop(); front(); back(); size();
  \end{lstlisting}

  A prioritásos sor (\kkod{priotity\textunderscore{}queue})
  \kkod{interface} minden standardnak, aminek van
  \kkod{back()}, \kkod{pusk\textunderscore{}back()}
  és \kkod{pop\textunderscore{}back()} művelete.
  (\kkod{vector}, \kkod{deque}, \kkod{list})
  \begin{lstlisting}
    // queue
    #include <queue>

    queue<type, container<type>>;

    empty(); push(); pop(); front(); size();
  \end{lstlisting}

  \end{document}