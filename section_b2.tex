\documentclass[main.tex]{subfiles}

\begin{document}

  \section{Tétel – Az operációs rendszerek alapjai} %13
  
  \fogalom{Az operációs rendszer céljai, feladatai}
  lorem
  \fogalom{Folyamatok kommunikációja}
  lorem
  \fogalom{Termelő-fogyasztó probléma}
  lorem
  \fogalom{Postaláda kezelés}
  lorem
  \fogalom{Szemaforok}
  lorem


  \section{Tétel – Az operációs rendszerek alapjai} %14
  

  \fogalom{Holtpont}
  lorem
  \fogalom{Holtpont kezelése}
  lorem
  \fogalom{Holtpont észlelése}
  lorem
  \fogalom{Holtpont megelőzés}
  lorem
  \fogalom{Bankár algoritmus}
  lorem


  \section{Tétel – Az operációs rendszerek alapjai} %15
  
  \fogalom{Ütemezési algoritmusok}
  lorem
  \fogalom{Előbb jött – előbb fut algoritmus}
  lorem
  \fogalom{A legrövidebb előnyben algoritmus}
  lorem
  \fogalom{Körbenforgó algoritmus}
  lorem


  \section{Tétel – Az információelmélet alapjai} %16

  \fogalom{Shannon hírközlési modellje}
  lorem
  \fogalom{Az információ}
  lorem
  \fogalom{Az entrópia}
  lorem
  \fogalom{Forráskódok}
  lorem
  \fogalom{Egyértelműen dekódolható kódok}
  lorem
  \fogalom{Prefix kód.
  }
  lorem


  \section{Tétel – Az információelmélet alapjai} %16

  \fogalom{Forráskódolás}
  lorem
  \fogalom{Huffman-kód}
  lorem
  \fogalom{Csatornakódolás}
  lorem
  \fogalom{Hamming- távolság}
  lorem
  \fogalom{Hibajelzés}
  lorem



\end{document}