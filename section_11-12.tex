\documentclass[main.tex]{subfiles}

\begin{document}
  % ----------------------------------------------------------------------------
  % ------------------------------ Section 11 ----------------------------------
  % ----------------------------------------------------------------------------
  \section{Tétel – A számítógép architektúrák alapjai} %11

  \fogalom{Boole függvények}
  \begin{itemize}
    \item 2 változós \zkod{Boole függvény}ekből 16 darab van
    
    \item $n$ változósból $2^{2^n}$
    
    \item \kkod{AND}, \kkod{OR}, \kkod{NOT} függvényekből
    az összes Boole függvény előállítható

    \item a \kkod{NAND} és a \kkod{NOR} önmagukban képesek
    az összeset előállítani
  \end{itemize}

  \fogalom{Logikai kapuk}
  \begin{figure}[H]
    \centering
    \begin{circuitikz}[american, thick]
      \ctikzset{american or shape=roundy}

      \node [and port](myand) at (0,0) {};
      \node [below=4pt] (andnode) at (myand.south) {\kkod{AND}};

      \node [nand port](mynand) at (3,0) {};
      \node [below=4pt] (nandnode) at (mynand.south) {\kkod{NAND}};

      \node [or port](myor) at (6,0) {};
      \node [below=4pt] (ornode) at (myor.south) {\kkod{OR}};

      \node [nor port](mynor) at (9,0) {};
      \node [below=4pt] (nornode) at (mynor.south) {\kkod{NOR}};

      \node [xor port](myxor) at (12,0) {};
      \node [below=4pt] (xornode) at (myxor.south) {\kkod{XOR}};

      \node [not port](mynot) at (-3,0) {};
      \node [below=4pt] (notnode) at (mynot.south) {\kkod{NOT}};
    \end{circuitikz}
  \end{figure}
  
  \begin{minipage}[t]{0.166\textwidth}
    \begin{center}
      \begin{tabular}{|c|c|}
        \hline
        $in$ & $out$
        \\ \hline \hline
        0 & 1
        \\ \hline
        1 & 0
        \\ \hline
      \end{tabular}
    \end{center}
  \end{minipage}\hfill
  \begin{minipage}[t]{0.166\textwidth}
    \begin{center}
      \begin{tabular}{|c|c|c|}
        \hline
        $i_1$ & $i_2$ & $out$
        \\ \hline \hline
        0 & 0 & 0
        \\ \hline
        0 & 1 & 0
        \\ \hline
        1 & 0 & 0
        \\ \hline
        1 & 1 & 1
        \\ \hline
      \end{tabular}
    \end{center}
  \end{minipage}\hfill
  \begin{minipage}[t]{0.166\textwidth}
    \begin{center}
      \begin{tabular}{|c|c|c|}
        \hline
        $i_1$ & $i_2$ & $out$
        \\ \hline \hline
        0 & 0 & 1
        \\ \hline
        0 & 1 & 1
        \\ \hline
        1 & 0 & 1
        \\ \hline
        1 & 1 & 0
        \\ \hline
      \end{tabular}
    \end{center}
  \end{minipage}\hfill
  \begin{minipage}[t]{0.166\textwidth}
    \begin{center}
      \begin{tabular}{|c|c|c|}
        \hline
        $i_1$ & $i_2$ & $out$
        \\ \hline \hline
        0 & 0 & 0
        \\ \hline
        0 & 1 & 1
        \\ \hline
        1 & 0 & 1
        \\ \hline
        1 & 1 & 1
        \\ \hline
      \end{tabular}
    \end{center}
  \end{minipage}\hfill
  \begin{minipage}[t]{0.166\textwidth}
    \begin{center}
      \begin{tabular}{|c|c|c|}
        \hline
        $i_1$ & $i_2$ & $out$
        \\ \hline \hline
        0 & 0 & 1
        \\ \hline
        0 & 1 & 0
        \\ \hline
        1 & 0 & 0
        \\ \hline
        1 & 1 & 0
        \\ \hline
      \end{tabular}
    \end{center}
  \end{minipage}\hfill
  \begin{minipage}[t]{0.166\textwidth}
    \begin{center}
      \begin{tabular}{|c|c|c|}
        \hline
        $i_1$ & $i_2$ & $out$
        \\ \hline \hline
        0 & 0 & 0
        \\ \hline
        0 & 1 & 1
        \\ \hline
        1 & 0 & 1
        \\ \hline
        1 & 1 & 0
        \\ \hline
      \end{tabular}
    \end{center}
  \end{minipage}\hfill
  
  \fogalom{Kombinációs logikai hálózatok}
  \zkod{Kombinációs logikai hálózatok} esetén a
  kimeneti jelek értékei csak a bemeneti jelek
  pillanatnyi értékétől függenek. A kimenetek
  egy-egy függvénykapcsolattal írhatóak le.

  \vspace{1em}
  {\large \zkod{Félösszeadó}:}

  \begin{minipage}[c]{0.3\textwidth}
    \begin{figure}[H]
      \centering
      \begin{circuitikz}[american, thick]
        \ctikzset{american or shape=roundy}
  
        \draw (0, 4)node[xor port] (xorone){}
        (0, 2)node[and port] (and){}
        (xorone.in 1) node[left=1cm](a) {A}
        (xorone.in 2) node[left=1cm](b) {B}
        (xorone.out) node[right=.5cm](s) {S}
        (and.out) node[right=.5cm](c) {C}
  
        (a.east) to[short,-*] (xorone.in 1) |- (and.in 1)
        (b.east) to[short,-*] ($(b.east)!.5!(xorone.in 2)$) coordinate (branch) -- (xorone.in 2)
        (c.west) to (and.out)
        (s.west) to (xorone.out)
        (branch) |- (and.in 2);  
      \end{circuitikz}
    \end{figure}
  \end{minipage}\hfill
  \begin{minipage}[c]{0.2\textwidth}
    \vspace{1.5em}
    \begin{center}
      \begin{tabular}{|c|c|c|c|}
        \hline
        $A$ & $B$ & $S$ & $C$
        \\ \hline \hline
        0 & 0 & 0 & 0
        \\ \hline
        0 & 1 & 1 & 0
        \\ \hline
        1 & 0 & 1 & 0
        \\ \hline
        1 & 1 & 0 & 1
        \\ \hline
      \end{tabular}
    \end{center}
  \end{minipage}\hfill
  \begin{minipage}[c]{0.5\textwidth}
    \begin{itemize}
      \vspace{1.5em}
      \item egy \kkod{XOR} és \kkod{AND} kapuval megvalósítható
      
      \item feladata 2 bit összeadása
      
      \item $S = \overline{A}B + A\overline{B}$
      \tabto{3.2cm} – \tabto{4cm} összeg
      
      \item $C = AB$
      \tabto{3.2cm} – \tabto{4cm} maradék (carry)
    \end{itemize}
  \end{minipage}\hfill

  \vspace{2em}
  {\large \zkod{Teljes összeadó}:}
  \begin{itemize}
    \item feladata két bit és az előző helyi
    értékből származó maradék összeadása

    \item \fkod{bemenetek:} $A$, $B$, $C_{\mathrm{in}}$,
    \fkod{kimenetek:} $S$, $C_{\mathrm{out}}$

    \item $S
    = \overline{A}\overline{B}{C}_\mathrm{in}
    + \overline{A}{B}\overline{C}_\mathrm{in}
    + {A}\overline{B}\overline{C}_\mathrm{in}
    + {A}{B}{C_\mathrm{in}}$

    \item $C
    = \overline{A}{B}{C}_\mathrm{in}
    + {A}\overline{B}{C}_\mathrm{in}
    + {A}{B}\overline{C}_\mathrm{in}
    + {A}{B}{C_\mathrm{in}}
    = AB + BC_\mathrm{in} + AC_\mathrm{in}$
  \end{itemize}

  \vspace{2em}
  {\large \zkod{Két négy bites szám összeadása}:}
  \begin{itemize}
    \item 4 teljes összeadó segítségével lehetséges
  \end{itemize}

  \pagebreak
  {\large \zkod{Multiplexer}:}

  \begin{minipage}[c]{0.2\textwidth}
    \begin{figure}[H]
      \centering
      \begin{circuitikz}[american, thick]
        \tikzset{mux 4by2/.style={muxdemux,
          muxdemux def={Lh=5, NL=4, Rh=3,
          NB=2, w=2, square pins=1}}}
        
          \draw (0,0) node [mux 4by2] (mux1) {}
          
          (mux1.lpin 1) node [left=6](in1) {$D_0$}
          (mux1.lpin 2) node [left=6](in2) {$D_1$}
          (mux1.lpin 3) node [left=6](in3) {$D_2$}
          (mux1.lpin 4) node [left=6](in4) {$D_3$}
          
          (mux1.rpin 1) node [right](y) {$Y$}
          
          (mux1.bpin 1) node [below](a) {$A$}
          (mux1.bpin 2) node [below](b) {$B$}

          (in1.east) to (mux1.lpin 1)
          (in2.east) to (mux1.lpin 2)
          (in3.east) to (mux1.lpin 3)
          (in4.east) to (mux1.lpin 4)

          (a.north) to (mux1.bpin 1)
          (b.north) to (mux1.bpin 2)

          (y.west) to (mux1.rpin 1);
      \end{circuitikz}
    \end{figure}
  \end{minipage}\hfill
  \begin{minipage}[c]{0.2\textwidth}
    \begin{center}
      \begin{tabular}{|c|c|c|}
        \hline
        $A$ & $B$ & $Y$
        \\ \hline \hline
        0 & 0 & $D_0$
        \\ \hline
        0 & 1 & $D_1$
        \\ \hline
        1 & 0 & $D_2$
        \\ \hline
        1 & 1 & $D_3$
        \\ \hline
      \end{tabular}
    \end{center}
  \end{minipage}\hfill
  \begin{minipage}[c]{0.6\textwidth}
    \begin{itemize}
      \item feladata több bemenő jel
      közül az egyik kiválasztása
      
      \item $2^n$ db \fkod{bemenet}\\
      $1$ db \fkod{kimenet}\\
      $n$ db \fkod{vezérlőbemenet}
      
      \item lehet még párhuzamos–soros adatkonverter
    \end{itemize}
  \end{minipage}\hfill
  
  \vspace{2em}
  {\large \zkod{Demultiplexer}:}

  \begin{minipage}[c]{0.2\textwidth}
    \begin{figure}[H]
      \centering
      \begin{circuitikz}[american, thick]
        \tikzset{demux 2by4/.style={muxdemux,
        muxdemux def={Lh=3, NL=1, Rh=5,
        NR=4, NB=2, w=2, square pins=1}}}
        
        \draw (0,0) node [demux 2by4] (demux1) {}
        
        (demux1.rpin 1) node [right=6](out1) {$D_0$}
        (demux1.rpin 2) node [right=6](out2) {$D_1$}
        (demux1.rpin 3) node [right=6](out3) {$D_2$}
        (demux1.rpin 4) node [right=6](out4) {$D_3$}
        
        (demux1.lpin 1) node [left](y) {$Y$}
        
        (demux1.bpin 1) node [below](outa) {$A$}
        (demux1.bpin 2) node [below](outb) {$B$}

        (out1.west) to (demux1.rpin 1)
        (out2.west) to (demux1.rpin 2)
        (out3.west) to (demux1.rpin 3)
        (out4.west) to (demux1.rpin 4)

        (outa.north) to (demux1.bpin 1)
        (outb.north) to (demux1.bpin 2)

        (y.east) to (demux1.lpin 1);
      \end{circuitikz}
    \end{figure}
  \end{minipage}\hfill
  \begin{minipage}[c]{0.2\textwidth}
    \begin{center}
      \begin{tabular}{|c|c|c|}
        \hline
        $A$ & $B$ & $Y$
        \\ \hline \hline
        0 & 0 & $D_0$
        \\ \hline
        0 & 1 & $D_1$
        \\ \hline
        1 & 0 & $D_2$
        \\ \hline
        1 & 1 & $D_3$
        \\ \hline
      \end{tabular}
    \end{center}
  \end{minipage}\hfill
  \begin{minipage}[c]{0.6\textwidth}
    \begin{itemize}
      \item egy jel kapcsolása választható kimenetre
      
      \item $1$ db \fkod{bemenet}\\
      $2^n$ db \fkod{kimenet}\\
      $n$ db \fkod{vezérlőbemenet}
      
      \item lehet még párhuzamos–soros adatkonverter
    \end{itemize}
  \end{minipage}\hfill


  \vspace{2em}
  {\large \zkod{Címdekóder}:}

  \begin{minipage}[c]{0.35\textwidth}
    \begin{figure}[H]
      \centering
      \begin{circuitikz}[american, thick]
        \tikzset{decoder/.style={muxdemux,
        muxdemux def={Lh=7, NL=3, Rh=7,
        NR=8, NB=0, w=3.5, square pins=1}}}
        
        \draw (0,0) node [decoder] (decoder1) {\small \kkod{Decoder}}
        
        (decoder1.lpin 1) node [left](in1) {$A$}
        (decoder1.lpin 2) node [left](in2) {$B$}
        (decoder1.lpin 3) node [left](in3) {$C$}
        
        (decoder1.rpin 1) node [right](out1) {\footnotesize $D_0$}
        (decoder1.rpin 2) node [right](out2) {\footnotesize $D_1$}
        (decoder1.rpin 3) node [right](out3) {\footnotesize $D_2$}
        (decoder1.rpin 4) node [right](out4) {\footnotesize $D_3$}
        (decoder1.rpin 5) node [right](out5) {\footnotesize $D_4$}
        (decoder1.rpin 6) node [right](out6) {\footnotesize $D_5$}
        (decoder1.rpin 7) node [right](out7) {\footnotesize $D_6$}
        (decoder1.rpin 8) node [right](out8) {\footnotesize $D_7$};
      \end{circuitikz}
    \end{figure}
  \end{minipage}\hfill
  \begin{minipage}[c]{0.25\textwidth}
    \begin{center}
      \begin{tabular}{|c|c|c|c|}
        \hline
        $A$ & $B$ & $C$ & $Y$
        \\ \hline \hline
        0 & 0 & 0 & $D_0$
        \\ \hline
        0 & 0 & 1 & $D_1$
        \\ \hline
        0 & 1 & 0 & $D_2$
        \\ \hline
        0 & 1 & 1 & $D_3$
        \\ \hline
        1 & 0 & 0 & $D_4$
        \\ \hline
        1 & 0 & 1 & $D_5$
        \\ \hline
        1 & 1 & 0 & $D_6$
        \\ \hline
        1 & 1 & 1 & $D_7$
        \\ \hline
      \end{tabular}
    \end{center}
  \end{minipage}\hfill
  \begin{minipage}[c]{0.4\textwidth}
    \begin{itemize}
      \item feladata cím dekódolása
      
      \item \fkod{bemenet:} $n$ bites szám
      
      \item \fkod{kimenet:} $2^n$-ből választ ki $1$-et
    \end{itemize}
  \end{minipage}\hfill


  \fogalom{Szekvenciális logikai hálózatok}
  \zkod{Szekvenciális logikai hálózatok} esetén
  a kimenet nemcsak a bemeneti jelkombinációtól,
  hanem a hálózat állapotától is függ.
  (azaz a a hálózatra megelőzően ható jelkombinációktól)
  Léteznek \zkod{szinkron} (órajel) és \zkod{aszinkron}
  sorrendi hálózatok.
  
  \fogalom{Flip-flop}
  \begin{minipage}[c]{0.25\textwidth}
    \begin{figure}[H]
      \centering
      \begin{circuitikz}[american, thick]
        \ctikzset{american or shape=roundy}

        \node[nor port] (nor1) at (0, 1.2) {};
        \node[nor port] (nor2) at (0,-1.2) {};

        \draw (nor1.in 2) -| ++ (-0.3,-0.5) -- ++(2.15,-0.8) coordinate(a)|- (nor2.out);
        \draw (nor2.in 1) -| ++ (-0.3, 0.5) -- ++(2.15, 0.8) |- (nor1.out);

        \draw (nor1.out -| a) to [short, *-] ++(0.45,0) node[right]{$Q$};
        \draw (nor2.out -| a) to [short, *-] ++(0.45,0) node[right]{$\overline{Q}$};

        \draw (nor1.in 1) -- ++ (-.5, 0) node[left]{$S$};
        \draw (nor2.in 2) -- ++ (-.5, 0) node[left]{$R$};
      \end{circuitikz}
    \end{figure}
  \end{minipage}\hfill
  \begin{minipage}[c]{0.25\textwidth}
    \begin{center}
      \begin{tabular}{|c|c|c|}
        \hline
        $S$ & $R$ & $Q$
        \\ \hline \hline
        0 & 0 & prev
        \\ \hline
        0 & 1 & 0
        \\ \hline
        1 & 0 & 1
        \\ \hline
        1 & 1 & ?
        \\ \hline
      \end{tabular}
    \end{center}
  \end{minipage}\hfill
  \begin{minipage}[c]{0.5\textwidth}
    \begin{itemize}
      \item elemi sorrendi hálózatok
      
      \item két stabil állapotú billenő elemek
      
      \item állapotuk megegyezik a kimenettel
      
      \item alkalmazásuk:
    \end{itemize}
  \end{minipage}\hfill


  % ----------------------------------------------------------------------------
  % ------------------------------ Section 12 ----------------------------------
  % ----------------------------------------------------------------------------
  \section{Tétel – A számítógép architektúrák alapjai} %12

  \fogalom{A számítógép felépítése}
  \begin{itemize}
    \item \zkod{Hardver} \tabto{2cm} – \tabto{2.6cm}
    elektromos áramkörök, mechanikus berendezések, \\
    \tabto{2.6cm}kábelek, csatlakozók, perifériák \\
    \tabto{2.6cm}önmagában nem működőképesek
    
    \item \zkod{Szoftver} \tabto{2cm} – \tabto{2.6cm}
    számítógépet működőképessé tevő
    programok és azok dokumentációi

    \item \zkod{Firmware} \tabto{2cm} – \tabto{2.6cm}
    célprogram, mikrokóddal írt, készülék spacifikus, \\
    \tabto{2.6cm}hardverbe ágyazott szoftver,
    gyakran \zkod{Flash ROM}
  \end{itemize}

  A \zkod{digitális számítógép} olyan gép,
  amely a neki címzett \zkod{utasítás}ok alapján
  problémákat old meg. Az utasítássorozatot,
  amely leírja, hogy hogyan oldjunk meg egy feladatot,
  \zkod{program}nak nevezünk.

  \vspace{1em}
  {\large \zkod{Gépi, nyelvi szintek}:}
  \begin{enumerate}
    \setcounter{enumi}{-1} 
    \item 
    \item b
    \item c
  \end{enumerate}

  
  
  \fogalom{Memóriák}
  lorem
  \fogalom{CPU részei}
  lorem
  \fogalom{Utasítás ciklus}
  lorem
  \fogalom{Szubrutinhívás}
  lorem
  \fogalom{Interrupt}
  lorem
  \fogalom{Közvetlen memória hozzáférés}
  lorem




\end{document}