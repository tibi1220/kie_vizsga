\documentclass[main.tex]{subfiles}

\begin{document}
  % ----------------------------------------------------------------------------
  % ------------------------------ Section 11 ----------------------------------
  % ----------------------------------------------------------------------------
  \section{Tétel – A számítógép architektúrák alapjai} %11

  \fogalom{Boole függvények}
  \begin{itemize}
    \item 2 változós \zkod{Boole függvény}ekből 16 darab van
    
    \item $n$ változósból $2^{2^n}$
    
    \item \kkod{AND}, \kkod{OR}, \kkod{NOT} függvényekből
    az összes Boole függvény előállítható

    \item a \kkod{NAND} és a \kkod{NOR} önmagukban képesek
    az összeset előállítani
  \end{itemize}

  \fogalom{Logikai kapuk}
  A \zkod{digitális áramkör}ök esetén az áramkör
  bármely pontján mérhető jeleknek csak 2 állapotát
  külünbüztetjük meg.
  A digitális áramköröket \zkod{logikai áramkör}ökkel
  modellezzük. Leírásukhoz \zkod{Boole algebrá}t használunk.
  A \zkod{logikai kapu}k a logikai áramkörök építőkockái,
  logikai alapműveleteket valósítanak meg. Ezek kombinációjával
  további áramköröket tudunk felépíteni.
  \begin{figure}[H]
    \centering
    \begin{circuitikz}[american, thick]
      \ctikzset{american or shape=roundy}

      \node [and port](myand) at (0,0) {};
      \node [below=4pt] (andnode) at (myand.south) {\kkod{AND}};

      \node [nand port](mynand) at (3,0) {};
      \node [below=4pt] (nandnode) at (mynand.south) {\kkod{NAND}};

      \node [or port](myor) at (6,0) {};
      \node [below=4pt] (ornode) at (myor.south) {\kkod{OR}};

      \node [nor port](mynor) at (9,0) {};
      \node [below=4pt] (nornode) at (mynor.south) {\kkod{NOR}};

      \node [xor port](myxor) at (12,0) {};
      \node [below=4pt] (xornode) at (myxor.south) {\kkod{XOR}};

      \node [not port](mynot) at (-3,0) {};
      \node [below=4pt] (notnode) at (mynot.south) {\kkod{NOT}};
    \end{circuitikz}
  \end{figure}
  
  \begin{minipage}[t]{0.166\textwidth}
    \begin{center}
      \begin{tabular}{|c|c|}
        \hline
        $in$ & $out$
        \\ \hline \hline
        0 & 1
        \\ \hline
        1 & 0
        \\ \hline
      \end{tabular}
    \end{center}
  \end{minipage}\hfill
  \begin{minipage}[t]{0.166\textwidth}
    \begin{center}
      \begin{tabular}{|c|c|c|}
        \hline
        $i_1$ & $i_2$ & $out$
        \\ \hline \hline
        0 & 0 & 0
        \\ \hline
        0 & 1 & 0
        \\ \hline
        1 & 0 & 0
        \\ \hline
        1 & 1 & 1
        \\ \hline
      \end{tabular}
    \end{center}
  \end{minipage}\hfill
  \begin{minipage}[t]{0.166\textwidth}
    \begin{center}
      \begin{tabular}{|c|c|c|}
        \hline
        $i_1$ & $i_2$ & $out$
        \\ \hline \hline
        0 & 0 & 1
        \\ \hline
        0 & 1 & 1
        \\ \hline
        1 & 0 & 1
        \\ \hline
        1 & 1 & 0
        \\ \hline
      \end{tabular}
    \end{center}
  \end{minipage}\hfill
  \begin{minipage}[t]{0.166\textwidth}
    \begin{center}
      \begin{tabular}{|c|c|c|}
        \hline
        $i_1$ & $i_2$ & $out$
        \\ \hline \hline
        0 & 0 & 0
        \\ \hline
        0 & 1 & 1
        \\ \hline
        1 & 0 & 1
        \\ \hline
        1 & 1 & 1
        \\ \hline
      \end{tabular}
    \end{center}
  \end{minipage}\hfill
  \begin{minipage}[t]{0.166\textwidth}
    \begin{center}
      \begin{tabular}{|c|c|c|}
        \hline
        $i_1$ & $i_2$ & $out$
        \\ \hline \hline
        0 & 0 & 1
        \\ \hline
        0 & 1 & 0
        \\ \hline
        1 & 0 & 0
        \\ \hline
        1 & 1 & 0
        \\ \hline
      \end{tabular}
    \end{center}
  \end{minipage}\hfill
  \begin{minipage}[t]{0.166\textwidth}
    \begin{center}
      \begin{tabular}{|c|c|c|}
        \hline
        $i_1$ & $i_2$ & $out$
        \\ \hline \hline
        0 & 0 & 0
        \\ \hline
        0 & 1 & 1
        \\ \hline
        1 & 0 & 1
        \\ \hline
        1 & 1 & 0
        \\ \hline
      \end{tabular}
    \end{center}
  \end{minipage}\hfill
  
  \fogalom{Kombinációs logikai hálózatok}
  \zkod{Kombinációs logikai hálózatok} esetén a
  kimeneti jelek értékei csak a bemeneti jelek
  pillanatnyi értékétől függenek. A kimenetek
  egy-egy függvénykapcsolattal írhatóak le.
 
  \vspace{1em}
  {\large \zkod{Félösszeadó}:}

  \begin{minipage}[c]{0.3\textwidth}
    \begin{figure}[H]
      \centering
      \begin{circuitikz}[american, thick]
        \ctikzset{american or shape=roundy}
  
        \draw (0, 4)node[xor port] (xorone){}
        (0, 2)node[and port] (and){}
        (xorone.in 1) node[left=1cm](a) {$A$}
        (xorone.in 2) node[left=1cm](b) {$B$}
        (xorone.out) node[right=.5cm](s) {$S$}
        (and.out) node[right=.5cm](c) {$C$}
  
        (a.east) to[short,-*] (xorone.in 1) |- (and.in 1)
        (b.east) to[short,-*] ($(b.east)!.5!(xorone.in 2)$) coordinate (branch) -- (xorone.in 2)
        (c.west) to (and.out)
        (s.west) to (xorone.out)
        (branch) |- (and.in 2);  
      \end{circuitikz}
    \end{figure}
  \end{minipage}\hfill
  \begin{minipage}[c]{0.2\textwidth}
    \vspace{1.5em}
    \begin{center}
      \begin{tabular}{|c|c|c|c|}
        \hline
        $A$ & $B$ & $S$ & $C$
        \\ \hline \hline
        0 & 0 & 0 & 0
        \\ \hline
        0 & 1 & 1 & 0
        \\ \hline
        1 & 0 & 1 & 0
        \\ \hline
        1 & 1 & 0 & 1
        \\ \hline
      \end{tabular}
    \end{center}
  \end{minipage}\hfill
  \begin{minipage}[c]{0.5\textwidth}
    \begin{itemize}
      \vspace{1.5em}
      \item egy \kkod{XOR} és \kkod{AND} kapuval megvalósítható
      
      \item feladata 2 bit összeadása
      
      \item $S = \overline{A}B + A\overline{B}$
      \tabto{3.2cm} – \tabto{4cm} összeg
      
      \item $C = AB$
      \tabto{3.2cm} – \tabto{4cm} maradék (carry)
    \end{itemize}
  \end{minipage}\hfill

  \vspace{2em}
  {\large \zkod{Teljes összeadó}:}
  \begin{itemize}
    \item feladata két bit és az előző helyi
    értékből származó maradék összeadása

    \item \fkod{bemenetek:} $A$, $B$, $C_{\mathrm{in}}$,
    \fkod{kimenetek:} $S$, $C_{\mathrm{out}}$

    \item $S
    = \overline{A}\overline{B}{C}_\mathrm{in}
    + \overline{A}{B}\overline{C}_\mathrm{in}
    + {A}\overline{B}\overline{C}_\mathrm{in}
    + {A}{B}{C_\mathrm{in}}$

    \item $C
    = \overline{A}{B}{C}_\mathrm{in}
    + {A}\overline{B}{C}_\mathrm{in}
    + {A}{B}\overline{C}_\mathrm{in}
    + {A}{B}{C_\mathrm{in}}
    = AB + BC_\mathrm{in} + AC_\mathrm{in}$
  \end{itemize}

  \vspace{2em}
  {\large \zkod{Multiplexer}:}

  \begin{minipage}[c]{0.2\textwidth}
    \begin{figure}[H]
      \centering
      \begin{circuitikz}[american, thick]
        \tikzset{mux 4by2/.style={muxdemux,
          muxdemux def={Lh=5, NL=4, Rh=3,
          NB=2, w=2, square pins=1}}}
        
          \draw (0,0) node [mux 4by2] (mux1) {}
          
          (mux1.lpin 1) node [left=6](in1) {$D_0$}
          (mux1.lpin 2) node [left=6](in2) {$D_1$}
          (mux1.lpin 3) node [left=6](in3) {$D_2$}
          (mux1.lpin 4) node [left=6](in4) {$D_3$}
          
          (mux1.rpin 1) node [right](y) {$Y$}
          
          (mux1.bpin 1) node [below](a) {$A$}
          (mux1.bpin 2) node [below](b) {$B$}

          (in1.east) to (mux1.lpin 1)
          (in2.east) to (mux1.lpin 2)
          (in3.east) to (mux1.lpin 3)
          (in4.east) to (mux1.lpin 4)

          (a.north) to (mux1.bpin 1)
          (b.north) to (mux1.bpin 2)

          (y.west) to (mux1.rpin 1);
      \end{circuitikz}
    \end{figure}
  \end{minipage}\hfill
  \begin{minipage}[c]{0.2\textwidth}
    \begin{center}
      \begin{tabular}{|c|c|c|}
        \hline
        $A$ & $B$ & $Y$
        \\ \hline \hline
        0 & 0 & $D_0$
        \\ \hline
        0 & 1 & $D_1$
        \\ \hline
        1 & 0 & $D_2$
        \\ \hline
        1 & 1 & $D_3$
        \\ \hline
      \end{tabular}
    \end{center}
  \end{minipage}\hfill
  \begin{minipage}[c]{0.6\textwidth}
    \begin{itemize}
      \item feladata több bemenő jel
      közül az egyik kiválasztása
      
      \item $2^n$ db \fkod{bemenet}\\
      $1$ db \fkod{kimenet}\\
      $n$ db \fkod{vezérlőbemenet}
      
      \item lehet még párhuzamos–soros adatkonverter
    \end{itemize}
  \end{minipage}\hfill
  
  \vspace{2em}
  {\large \zkod{Demultiplexer}:}

  \begin{minipage}[c]{0.2\textwidth}
    \begin{figure}[H]
      \centering
      \begin{circuitikz}[american, thick]
        \tikzset{demux 2by4/.style={muxdemux,
        muxdemux def={Lh=3, NL=1, Rh=5,
        NR=4, NB=2, w=2, square pins=1}}}
        
        \draw (0,0) node [demux 2by4] (demux1) {}
        
        (demux1.rpin 1) node [right=6](out1) {$D_0$}
        (demux1.rpin 2) node [right=6](out2) {$D_1$}
        (demux1.rpin 3) node [right=6](out3) {$D_2$}
        (demux1.rpin 4) node [right=6](out4) {$D_3$}
        
        (demux1.lpin 1) node [left](y) {$Y$}
        
        (demux1.bpin 1) node [below](outa) {$A$}
        (demux1.bpin 2) node [below](outb) {$B$}

        (out1.west) to (demux1.rpin 1)
        (out2.west) to (demux1.rpin 2)
        (out3.west) to (demux1.rpin 3)
        (out4.west) to (demux1.rpin 4)

        (outa.north) to (demux1.bpin 1)
        (outb.north) to (demux1.bpin 2)

        (y.east) to (demux1.lpin 1);
      \end{circuitikz}
    \end{figure}
  \end{minipage}\hfill
  \begin{minipage}[c]{0.2\textwidth}
    \begin{center}
      \begin{tabular}{|c|c|c|}
        \hline
        $A$ & $B$ & $Y$
        \\ \hline \hline
        0 & 0 & $D_0$
        \\ \hline
        0 & 1 & $D_1$
        \\ \hline
        1 & 0 & $D_2$
        \\ \hline
        1 & 1 & $D_3$
        \\ \hline
      \end{tabular}
    \end{center}
  \end{minipage}\hfill
  \begin{minipage}[c]{0.6\textwidth}
    \begin{itemize}
      \item egy jel kapcsolása választható kimenetre
      
      \item $1$ db \fkod{bemenet}\\
      $2^n$ db \fkod{kimenet}\\
      $n$ db \fkod{vezérlőbemenet}
      
      \item lehet még párhuzamos–soros adatkonverter
    \end{itemize}
  \end{minipage}\hfill


  \vspace{2em}
  {\large \zkod{Címdekóder}:}

  \begin{minipage}[c]{0.35\textwidth}
    \begin{figure}[H]
      \centering
      \begin{circuitikz}[american, thick]
        \tikzset{decoder/.style={muxdemux,
        muxdemux def={Lh=7, NL=3, Rh=7,
        NR=8, NB=0, w=3.5, square pins=1}}}
        
        \draw (0,0) node [decoder] (decoder1) {\small \kkod{Decoder}}
        
        (decoder1.lpin 1) node [left](in1) {$A$}
        (decoder1.lpin 2) node [left](in2) {$B$}
        (decoder1.lpin 3) node [left](in3) {$C$}
        
        (decoder1.rpin 1) node [right](out1) {\footnotesize $D_0$}
        (decoder1.rpin 2) node [right](out2) {\footnotesize $D_1$}
        (decoder1.rpin 3) node [right](out3) {\footnotesize $D_2$}
        (decoder1.rpin 4) node [right](out4) {\footnotesize $D_3$}
        (decoder1.rpin 5) node [right](out5) {\footnotesize $D_4$}
        (decoder1.rpin 6) node [right](out6) {\footnotesize $D_5$}
        (decoder1.rpin 7) node [right](out7) {\footnotesize $D_6$}
        (decoder1.rpin 8) node [right](out8) {\footnotesize $D_7$};
      \end{circuitikz}
    \end{figure}
  \end{minipage}\hfill
  \begin{minipage}[c]{0.25\textwidth}
    \begin{center}
      \begin{tabular}{|c|c|c|c|}
        \hline
        $A$ & $B$ & $C$ & $Y$
        \\ \hline \hline
        0 & 0 & 0 & $D_0$
        \\ \hline
        0 & 0 & 1 & $D_1$
        \\ \hline
        0 & 1 & 0 & $D_2$
        \\ \hline
        0 & 1 & 1 & $D_3$
        \\ \hline
        1 & 0 & 0 & $D_4$
        \\ \hline
        1 & 0 & 1 & $D_5$
        \\ \hline
        1 & 1 & 0 & $D_6$
        \\ \hline
        1 & 1 & 1 & $D_7$
        \\ \hline
      \end{tabular}
    \end{center}
  \end{minipage}\hfill
  \begin{minipage}[c]{0.4\textwidth}
    \begin{itemize}
      \item feladata cím dekódolása
      
      \item \fkod{bemenet:} $n$ bites szám
      
      \item \fkod{kimenet:} $2^n$-ből választ ki $1$-et
    \end{itemize}
  \end{minipage}\hfill


  \fogalom{Szekvenciális logikai hálózatok}
  \zkod{Szekvenciális logikai hálózatok} esetén
  a kimenet nemcsak a bemeneti jelkombinációtól,
  hanem a hálózat állapotától is függ.
  (azaz a a hálózatra megelőzően ható jelkombinációktól)
  Léteznek \zkod{szinkron} (órajel) és \zkod{aszinkron}
  sorrendi hálózatok.
  
  \fogalom{Flip-flop}
  \begin{minipage}[c]{0.25\textwidth}
    \begin{figure}[H]
      \centering
      \begin{circuitikz}[american, thick]
        \ctikzset{american or shape=roundy}

        \node[nor port] (nor1) at (0, 1.2) {};
        \node[nor port] (nor2) at (0,-1.2) {};

        \draw (nor1.in 2) -| ++ (-0.3,-0.5) -- ++(2.15,-0.8) coordinate(a)|- (nor2.out);
        \draw (nor2.in 1) -| ++ (-0.3, 0.5) -- ++(2.15, 0.8) |- (nor1.out);

        \draw (nor1.out -| a) to [short, *-] ++(0.45,0) node[right]{$Q$};
        \draw (nor2.out -| a) to [short, *-] ++(0.45,0) node[right]{$\overline{Q}$};

        \draw (nor1.in 1) -- ++ (-.5, 0) node[left]{$S$};
        \draw (nor2.in 2) -- ++ (-.5, 0) node[left]{$R$};
      \end{circuitikz}
    \end{figure}
  \end{minipage}\hfill
  \begin{minipage}[c]{0.25\textwidth}
    \begin{center}
      \begin{tabular}{|c|c|c|}
        \hline
        $S$ & $R$ & $Q$
        \\ \hline \hline
        0 & 0 & prev
        \\ \hline
        0 & 1 & 0
        \\ \hline
        1 & 0 & 1
        \\ \hline
        1 & 1 & ?
        \\ \hline
      \end{tabular}
    \end{center}
  \end{minipage}\hfill
  \begin{minipage}[c]{0.5\textwidth}
    \begin{itemize}
      \item elemi sorrendi hálózatok
      
      \item két stabil állapotú billenő elemek
      
      \item állapotuk megegyezik a kimenettel
      
      \item regiszterek, \zkod{SRAM}, számlálók
    \end{itemize}
  \end{minipage}\hfill


  % ----------------------------------------------------------------------------
  % ------------------------------ Section 12 ----------------------------------
  % ----------------------------------------------------------------------------
  \section{Tétel – A számítógép architektúrák alapjai} %12

  \fogalom{A számítógép felépítése}
  \begin{itemize}
    \item \zkod{Hardver} \tabto{2cm} – \tabto{2.6cm}
    elektromos áramkörök, mechanikus berendezések, \\
    \tabto{2.6cm}kábelek, csatlakozók, perifériák \\
    \tabto{2.6cm}önmagában nem működőképesek
    
    \item \zkod{Szoftver} \tabto{2cm} – \tabto{2.6cm}
    számítógépet működőképessé tevő
    programok és azok dokumentációi

    \item \zkod{Firmware} \tabto{2cm} – \tabto{2.6cm}
    célprogram, mikrokóddal írt, készülék spacifikus, \\
    \tabto{2.6cm}hardverbe ágyazott szoftver,
    gyakran \zkod{Flash ROM}
  \end{itemize}

  A \zkod{digitális számítógép} olyan gép,
  amely a neki címzett \zkod{utasítás}ok alapján
  problémákat old meg. Az utasítássorozatot,
  amely leírja, hogy hogyan oldjunk meg egy feladatot,
  \zkod{program}nak nevezünk.

  \vspace{1em}
  {\large \zkod{Gépi, nyelvi szintek}:}
  \begin{enumerate}
    \setcounter{enumi}{-1} 
    \item digitális logikai szint
    \begin{itemize}
      \item kapuk (\zkod{gate})
    \end{itemize}
    
    \item mikroarchitektúra  szintje
    \begin{itemize}
      \item értelmezi a másofik szintet
      \item \zkod{ALU}, regiszterek
    \end{itemize}
    
    \item gépi utasítás szintje (elektronikus áramkörök)
    \begin{itemize}
      \item itt dől el a kompatibilitás kérdése
    \end{itemize}
    
    \item operációs rendszer gépi szintje
    \begin{itemize}
      \item általában értelmezésű
      \item az utasításait az oprendszer,
      vagy közvetlen a 2. szint hajtja végre
    \end{itemize}
    
    \item assembly nyelvi szint (assembler)
    \begin{itemize}
      \item szimbolikus leírás
    \end{itemize}
    
    \item probléma orientált nyelvi szint (fordító program)
    \begin{itemize}
      \item ezek tényleges nyelvek (\kkod{C}, \kkod{C++})
    \end{itemize}
  \end{enumerate}

  {\large \zkod{Neumann elvű számítógép felépítése}:}
  \vspace{1em}

  \begin{minipage}[c]{0.25\textwidth}
    \begin{figure}[H]
      \centering
      \begin{tikzpicture}
        
        \draw[thick] (0,0) rectangle ++(4,.75);
        \draw[thick] (0,-1.5) rectangle ++(4,.75);
        \draw[] (0,-1.5) rectangle ++(2,.75);
        \draw[thick] (0,-3) rectangle ++(4,.75);
  
        \draw[implies-implies,double equal sign distance, semithick] (2,0) -- (2,-0.75);
        \draw[implies-implies,double equal sign distance, semithick] (2,-1.5) -- (2,-2.25);
        
        \node[] at (2, .375) {\zkod{háttértárolók}};
        \node[] at (1, -1.125) {\zkod{CPU}};
        \node[] at (3, -1.125) {\zkod{memória}};
        \node[] at (2, -2.625) {\zkod{I/O eszközök}};
      \end{tikzpicture}
    \end{figure}
  \end{minipage}\hfill
  \begin{minipage}[c]{0.75\textwidth}
    \begin{itemize}
      \item a \zkod{CPU}
      általános vezérlő, műveletvégző, adat-mozgató egység,
      végrehajtja a futó programok utasításait

      \item a \zkod{memória}
      a futó programok kódját, adatait tartalmazza

      \item a \zkod{háttértárolók}
      lehet mágneslemez, merevlemez, optikai tároló,
      szalagos tároló, félvezezős tároló
      (flash memória chip)

      \item a \zkod{perifériák}:
      monitor, billentyűzet, egér,
      nyomtató, kommunikációs vonalak, stb.
    \end{itemize}
  \end{minipage}\hfill
  
  \pagebreak
  {\large \zkod{Számítógépek szokásos felépítése}:}
  \begin{itemize}
    \item a \zkod{részegységek} egy rendszersínen
    (\zkod{rendszerbusz}) keresztül kapcsolódnak egymáshoz

    \item tipikusan a renszerbusz, mikroprocesszor,
    memória, valamint az eszközvezérlők nagy része
    az alaplapon helyezkedik el

    \item bővítőkártyák is tartalmazhatnak eszközvezérlőket
    
    \item az eszközvezérlő képes lehet \zkod{DMA}-t végezni;
    ha kész, megszakítást vált ki

    \item 3 típusú információ áramolhat:
    \zkod{cím}, \zkod{adat}, \zkod{vezérlő}
  \end{itemize}

  {\large \zkod{Buszok}:}
  \begin{itemize}
    \item a \zkod{busz}ok jellemzésére
    az adat- és címvonalak számáz,
    az adatátvitel jellemzőit, időzítés adatait,
    a vezérlőjelek típusait, funkcióit kell megadni

    \item a \zkod{cím} lehet memóriacím, vagy
    \zkod{IO} eszköz címezhető

    \item a \zkod{vezérlőjel}ek lehetnek\dots
    \begin{itemize}
      \item adatátvitelt vezérlő jelek
      \begin{itemize}
        \item[$\circ$] cím a sínen
        \tabto{2.7cm} – \tabto{3.3cm}
        memória, periféria (\zkod{M}/\zkod{IO})

        \item[$\circ$] adat a sínen
        \tabto{2.7cm} – \tabto{3.3cm}
        írás, olvasás (\zkod{R}/\zkod{W})

        \item[$\circ$] átvitel vége
        \tabto{2.7cm} – \tabto{3.3cm}
        szó, byte átvitel (\zkod{WD}/\zkod{B})
      \end{itemize}

      \item megszakítást vezérlő jelek
      \item sínvezérlő jelek (kérés, foglalás, visszaigazolás)
      \item egyéb (órajel, ütemezés, táp)
    \end{itemize}
  \end{itemize}

  
  \fogalom{Memóriák}
  {\large \zkod{Memória hieracrhia}:}
  \begin{itemize}
    \item
    regiszter \\
    gyorsítótár \\
    központi memória \\
    mágneslemez \\
    szalag, optikai lemez
  \end{itemize}

  {\large \zkod{Csoportosítás}:}
  \begin{enumerate}
    \item információ elérése alapjám
    \begin{itemize}
      \item cím szerinti hozzáférés
      
      \item tartalom szerinti hozzáférés (cache)
    \end{itemize}

    \item hozzáférés belső szervezés alapján
    \begin{itemize}
      \item szekvenciális Memóriák
      
      \item tetszőleges sorrendben címezhető memóriák
      \begin{itemize}
        \item csak olvasható memóriák (\zkod{ROM})
        
        \item írható-olvasható memóriák (\zkod{RAM})
      \end{itemize}
    \end{itemize}
  \end{enumerate}

  {\large \zkod{Tetszőleges sorrendben címezhető memóriák}:}
  \begin{itemize}
    \item sor és oszlopdekóderek
    \item író, olvasó
    \item memóriacella egy bit tárolására képes
  \end{itemize}

  {\large \zkod{ROM típusok}:}

  Minden egyes típus egyedi karakterisztikával bír,
  de két dologban közösek. Az \zkod{eltárolt adatok}
  ezekben a lapkákban \zkod{nem illékonyak}, azaz nem
  vesznek el, amikor kikapcsoljuk az áramot.
  Az eltárolt adatok \zkod{megváltoztathatatlanok},
  vagy speciális műveletet igényel a változtatás.
  (Ellentétben a \zkod{RAM}-mal, melynél könnyű a változtatás)

  \begin{itemize}
    \item \zkod{ROM} \tabto{1.8cm} – \tabto{2.5cm}
    Read Only Memory
    \begin{itemize}
      \item a gyártó programozza
    \end{itemize}
    
    \item \zkod{PROM} \tabto{1.8cm} – \tabto{2.5cm}
    Programmable Read Only Memory
    \begin{itemize}
      \item felhasználó egyszer programozhatja,
      azaz megfelelő készülékkel kiégetheti 
      a cellákban lévő tranzisztorok bekötő vezetékeit
    \end{itemize}
    
    \item \zkod{EPROM} \tabto{1.8cm} – \tabto{2.5cm}
    Erasable Programmable Read Only Memory
    \begin{itemize}
      \item UV fénnyel törölhető, majd külön
      készülékkel újra írható a tartalma

      \item régebben a \zkod{ROM BIOS} ilyen
      memóriában helyezkedett el
    \end{itemize}

    \item \zkod{EEPROM} \tabto{1.8cm} – \tabto{2.5cm}
    Electrically Erasable Programmable Read Only Memory
    \begin{itemize}
      \item elektromosan törölhető,majd külön
      készülékkel újra írható a tartalma
    \end{itemize}

    \item \zkod{FLASH} \tabto{1.8cm} – \tabto{2.5cm}
    Flash/Villanó Memória
    \begin{itemize}
      \item Olyan \zkod{EEPROM}, melyet számítógép
      is képes törölni, majd újreírni.

      \item Pendrive-okban, fényképezőgépekben
    \end{itemize}
  \end{itemize}

  {\large \zkod{RAM típusok}:}
  \begin{itemize}
    \item \zkod{SRAM} \tabto{1.8cm} – \tabto{2.5cm}
    Static Random Access Memory
    \begin{itemize}
      \item  a tápfeszültség biztosításával korlátlan
      ideig megőrzi az információt

      \item a memóriacellában egy \zkod{flip-flop} található
      
      \item kisebb integráltságú
      (nagyobb méretű egy cella, mint a \zkod{DRAM} esetén)

      \item nagyon gyors: \zkod{cache}
    \end{itemize}

    \item \zkod{DRAM} \tabto{1.8cm} – \tabto{2.5cm}
    Dynamic Random Access Memory
    \begin{itemize}
      \item az információt egy pici
      \zkod{kondenzátor} tárolja

      \item a szivárgás miatt rövid időn belül elveszítené a töltését,
      ezért időközönként (néhány ms) frissíteni kell a tartalmát

      \item nagy integráltságú, a PC-k memóriája ilyen
    \end{itemize}
  \end{itemize}


  \fogalom{CPU részei}
  A \zkod{CPU} (Central Processing Unit – központi feldolgozó egység)
  a \zkod{memóriá}ból olvassa a végrehajtás alatt lévő
  \zkod{program bináris utasítása}it.
  Az \zkod{utasításkészlet}e fontos jellemzője.
  A \zkod{mikroprocesszor} egy chipen kialakított áramkör,
  mely a számítógép \zkod{CPU}-jának a funkcióját látja el.
  Részei:
  \begin{itemize}
    \item \zkod{ALU} – aritmetikai és logikai műveletek végzése
    \begin{itemize}
      \item összeadás, kivonás,
      fixpontos szorzás, osztás (léptetések),
      lebegőpontos aritmetikai műveletek (korábban koprocesszor),
      egyszerű logikai műveletek
    \end{itemize}

    \item \zkod{utasítás dekódoló és vezérlő egység}
    \begin{itemize}
      \item Felismeri, elemzi (dekódolja) a gépi nyelvű
      program utasításait, az utasítások alapján működteti
      a \zkod{CPU} többi egységét, illetve képezi a szükséges címeket.
    \end{itemize}

    \item \zkod{regiszterek} –
    chipen belüli, közvetlen elérésű tároló elemek
    \begin{itemize}
      \item feladatuk műveletvégzéskor az operandusok tárolása,
      illetve a címek előállítása
    \end{itemize}
  \end{itemize}

  {\large \zkod{8086 processzor}:}
  \begin{itemize}
    \item \zkod{szegmensregiszterek}
    \begin{itemize}
      \item \zkod{CS} – Code Segment
      \tabto{4cm} – \tabto{4.6cm}
      kódszegmens regiszter
      
      \item \zkod{SS} – Stack Segment
      \tabto{4cm} – \tabto{4.6cm}
      veremszegmens regiszter

      \item \zkod{DS} – Data Segment
      \tabto{4cm} – \tabto{4.6cm}
      adatszegmens regiszter

      \item \zkod{ES} – Extra Segment
      \tabto{4cm} – \tabto{4.6cm}
      extra adatszegmens regiszter
    \end{itemize}

    \item \zkod{vezérlő regiszterek}
    \begin{itemize}
      \item \zkod{IP} – Instruction Pointer
      \tabto{4.8cm} – \tabto{5.4cm}
      utasítás mutató
      
      \item \zkod{SP} – Stack Pointer
      \tabto{4.8cm} – \tabto{5.4cm}
      verem mutató

      \item \zkod{BP} – Base Pointer
      \tabto{4.8cm} – \tabto{5.4cm}
      bázis mutató

      \item \zkod{SI} – Source Index
      \tabto{4.8cm} – \tabto{5.4cm}
      forrás index

      \item \zkod{DI} – Destination Index
      \tabto{4.8cm} – \tabto{5.4cm}
      cél index
    \end{itemize}

    \item \zkod{általános célú regiszterek – adatregiszterek}
    \begin{itemize}
      \item \zkod{AX} – akkumulátor regiszter
      (\zkod{AH}, \zkod{AL})
      
      \item \zkod{BX} – bázis regiszter
      (\zkod{BH}, \zkod{BL})
      
      \item \zkod{CX} – számláló regiszter
      (\zkod{CH}, \zkod{CL})
      
      \item \zkod{DX} – adatregiszter
      (\zkod{DH}, \zkod{DL})

      \item műveletvégzéskor az operandusok tárolására
    \end{itemize}

    \item \zkod{flag}-ek – jelzőbitek, melyek\dots
    \begin{itemize}
      \item vagy a lehutóbb elvégzett aritmetikai
      műveletek eredményétől függően vesznek fel értékeket,
      vagy az processzor állapotára utalnak

      \item a \zkod{feltételes ugró} utasítások
      a \zkod{flag}-eket használják feltételre

      \item \zkod{aritmetikai flag}-ek:
      előjel flag (\zkod{sign}),
      zéró flag (\zkod{zero}),
      paritás flag (\zkod{parity}),
      átvitel flag (\zkod{carry})
      (legmagasabb helyiértéken képződött maradék)

      \item processzor állapotára utalóak:
      \zkod{trap} flag (program utasításonkénti végrehajtása),
      \zkod{interrupt} flag (megszakítás,
      a hardver egységek felől érkező megszakítások
      eljutnak-e a processzorhoz),
      \zkod{overflow} flag (túlcsordulás),
    \end{itemize}

    \item \zkod{CPU} címzése:
    \begin{itemize}
      \item \zkod{memóriacímek}:
      a program utasításainak beolvasására,
      adatainak írására, olvasására

      \item \zkod{IO címek}:
      a perifériákkal való kommunikációra
    \end{itemize}

    \item \zkod{utasításkészlet}
    \begin{itemize}
      \item mikroprocesszorok egyik legfontosabb jellemzője,
      hogy milyen utasításokat ismernek, milyen a gépi nyelvük

      \item a gépi utasítások \zkod{bináris jelsorozatok}
      
      \item az \zkod{assembly} nyelv a gépi utasításokat
      \zkod{mnemoni}kkal helyettesíti
    \end{itemize}

    \item \zkod{assembly} utasítások
    \begin{table}[H]
      \centering
      \begin{tabular}{|l r|}
        \hline
        \kkod{MOV} & \hspace*{1em}adatmozgatás \\
        \kkod{ADD} & összeadás \\
        \kkod{SUB} & kivonás \\
        \kkod{MUL} & szorzás${}^{1}$ \\
        \kkod{DIV} & osztás${}^{1}$ \\
        \hline
      \end{tabular}
      \begin{tabular}{|l r|}
        \hline
        \kkod{JMP} \fkod{flag} & ugrás \\
        \kkod{JZ} \fkod{flag} & ugrás${}^{2}$ \\
        \kkod{CMP} & összehasonlítás \\
        \kkod{PUSH}, \kkod{POP} & verem \\
        \kkod{LDA}, \kkod{STA} & \hspace*{1em}akkumulátor${}^{3}$ \\
        \hline
      \end{tabular}
    \end{table}
    \begin{enumerate}
      \item \kkod{MUL} $\rightarrow$
      \fkod{AL}$\cdot$\fkod{XX}$=$\fkod{AX}
      \hspace{2em}
      \kkod{DIV} $\rightarrow$
      \fkod{AX}$/$\fkod{XX}$=$\fkod{AL}

      \item \kkod{JZ} $\rightarrow$
      akkor ugrik ha a \fkod{Zero} flag aktív

      \item \kkod{LDA} $\rightarrow$
      2 byte-ot másol a memóriából az akkumulátorba
      (\zkod{LoaD Accumulator})
      \\
      \kkod{STA} $\rightarrow$
      az akkumulátor tartalmát a memóriába másolja
      (\zkod{STore Accumulator})
    \end{enumerate}
  \end{itemize}


  \fogalom{Utasítás ciklus}
  \begin{enumerate}
    \item \zkod{fetch} (elérés)
    \begin{itemize}
      \item utasítás kód beolvasása
      \item utasítás kód értelmezése (dekódolás)
      \item operandusok beolvasása
    \end{itemize}
    \item \zkod{execute} (végrehajtás)
    \begin{itemize}
      \item műveletvégzés (\zkod{ALU})
      \item eredmény tárolása
      \item következő utasítás címének kiszámítása
    \end{itemize}
  \end{enumerate}
  
  \fogalom{Szubrutinhívás}
  \zkod{Szubrutinhívás} esetén a program máshol folytatódik.
  (alprogramra ugrunk)
  \kkod{CALL}, \kkod{RET} utasításpár.
  ahhoz, hogy vissza tudjunk térni a megfelelő
  helyre való visszatérést, ahhoz el kell
  menteni a \zkod{PC} (utasításszámláló) értékét a
  \kkod{Stack}-be (\kkod{PUSH})

  \fogalom{Interrupt}
  A megszakítás (\zkod{interrupt}) egy erőltetett vezérlésátadás,
  ugrás egy megszakítást kezelő rutinra. Előidézheti
  egy mikroprocesszorban előforduló esemény. (zéróosztás)
  Érkezhet megszakítás egy hardver egység felől is.
  (adatok beolvasása a memóriába megtörtént)
  Program is tartalmazhat megszakítási utasítást.
  (oprendszeri szolgáltatás) Fő okai:
  \begin{itemize}
    \item processzor megszakítás
    \item hardver megszakítás (\zkod{IRQ}: interrupt request)
    (lehet maszkolható, vagy nem maszkolható)
    \item szoftveres megszakítás
  \end{itemize}
  A \zkod{szubrutinhívás}hoz hasonlóan
  az utasításszámláló értékét a \kkod{stack}-be mentjük.

  \fogalom{Közvetlen memória hozzáférés}
  A \zkod{DMA} (Direct Memory Access) a memória
  és egy periféria (merevlemez) közötti közvetlen
  adatátvitel. A \zkod{DMA vezérkő} irányítja az
  adatforgalmat, igy a \zkod{CPU} közben egy
  másik program kódját futtatja. \zkod{DMA} nélkül
  az adatokat a processzoran kellene átvezetni,
  amely nagyon időigényes lenne. Kezdetben szükség
  van az iniciálásra, de utána a folyamat a processzor
  igénybevétele nélkül folytatódik.




\end{document}