\documentclass[main.tex]{subfiles}

\begin{document}
  % ----------------------------------------------------------------------------
  % ------------------------------ Section 11 ----------------------------------
  % ----------------------------------------------------------------------------
  \section{Tétel – A számítógép architektúrák alapjai} %11

  \fogalom{Boole függvények}
  \begin{itemize}
    \item 2 változós \zkod{Boole függvény}ekből 16 darab van
    
    \item $n$ változósból $2^{2^n}$
    
    \item \kkod{AND}, \kkod{OR}, \kkod{NOT} függvényekből
    az összes Boole függvény előállítható

    \item a \kkod{NAND} és a \kkod{NOR} önmagukban képesek
    az összeset előállítani
  \end{itemize}

  \fogalom{Logikai kapuk}
  \begin{figure}[H]
    \centering
    \begin{circuitikz}[american]
      \ctikzset{american or shape=roundy}

      \node [and port](myand) at (0,0) {};
      \node [below=4pt] (andnode) at (myand.south) {\kkod{AND}};

      \node [nand port](mynand) at (3,0) {};
      \node [below=4pt] (nandnode) at (mynand.south) {\kkod{NAND}};

      \node [or port](myor) at (6,0) {};
      \node [below=4pt] (ornode) at (myor.south) {\kkod{OR}};

      \node [nor port](mynor) at (9,0) {};
      \node [below=4pt] (nornode) at (mynor.south) {\kkod{NOR}};

      \node [xor port](myxor) at (12,0) {};
      \node [below=4pt] (xornode) at (myxor.south) {\kkod{XOR}};

      \node [not port](mynot) at (-3,0) {};
      \node [below=4pt] (notnode) at (mynot.south) {\kkod{NOT}};
    \end{circuitikz}
  \end{figure}
  
  \begin{minipage}[t]{0.166\textwidth}
    \begin{center}
      \begin{tabular}{|c|c|}
        \hline
        $in$ & $out$
        \\ \hline \hline
        0 & 1
        \\ \hline
        1 & 0
        \\ \hline
      \end{tabular}
    \end{center}
  \end{minipage}\hfill
  \begin{minipage}[t]{0.166\textwidth}
    \begin{center}
      \begin{tabular}{|c|c|c|}
        \hline
        $i_1$ & $i_2$ & $out$
        \\ \hline \hline
        0 & 0 & 0
        \\ \hline
        0 & 1 & 0
        \\ \hline
        1 & 0 & 0
        \\ \hline
        1 & 1 & 1
        \\ \hline
      \end{tabular}
    \end{center}
  \end{minipage}\hfill
  \begin{minipage}[t]{0.166\textwidth}
    \begin{center}
      \begin{tabular}{|c|c|c|}
        \hline
        $i_1$ & $i_2$ & $out$
        \\ \hline \hline
        0 & 0 & 1
        \\ \hline
        0 & 1 & 1
        \\ \hline
        1 & 0 & 1
        \\ \hline
        1 & 1 & 0
        \\ \hline
      \end{tabular}
    \end{center}
  \end{minipage}\hfill
  \begin{minipage}[t]{0.166\textwidth}
    \begin{center}
      \begin{tabular}{|c|c|c|}
        \hline
        $i_1$ & $i_2$ & $out$
        \\ \hline \hline
        0 & 0 & 0
        \\ \hline
        0 & 1 & 1
        \\ \hline
        1 & 0 & 1
        \\ \hline
        1 & 1 & 1
        \\ \hline
      \end{tabular}
    \end{center}
  \end{minipage}\hfill
  \begin{minipage}[t]{0.166\textwidth}
    \begin{center}
      \begin{tabular}{|c|c|c|}
        \hline
        $i_1$ & $i_2$ & $out$
        \\ \hline \hline
        0 & 0 & 1
        \\ \hline
        0 & 1 & 0
        \\ \hline
        1 & 0 & 0
        \\ \hline
        1 & 1 & 0
        \\ \hline
      \end{tabular}
    \end{center}
  \end{minipage}\hfill
  \begin{minipage}[t]{0.166\textwidth}
    \begin{center}
      \begin{tabular}{|c|c|c|}
        \hline
        $i_1$ & $i_2$ & $out$
        \\ \hline \hline
        0 & 0 & 0
        \\ \hline
        0 & 1 & 1
        \\ \hline
        1 & 0 & 1
        \\ \hline
        1 & 1 & 0
        \\ \hline
      \end{tabular}
    \end{center}
  \end{minipage}\hfill
  
  \fogalom{Kombinációs logikai hálózatok}
  \zkod{Kombinációs logikai hálózatok} esetén a
  kimeneti jelek értékei csak a bemeneti jelek
  pillanatnyi értékétől függenek. A kimenetek
  egy-egy függvénykapcsolattal írhatóak le.

  \vspace{1em}
  {\large \zkod{Félösszeadó}:}

  \begin{minipage}[c]{0.3\textwidth}
    \begin{figure}[H]
      \centering
      \begin{circuitikz}[american]
        \ctikzset{american or shape=roundy}
  
        \draw (0, 4)node[xor port] (xorone){}
        (0, 2)node[and port] (and){}
        (xorone.in 1) node[left=1cm](a) {A}
        (xorone.in 2) node[left=1cm](b) {B}
        (xorone.out) node[right=.5cm](s) {S}
        (and.out) node[right=.5cm](c) {C}
  
        (a.east) to[short,-*] (xorone.in 1) |- (and.in 1)
        (b.east) to[short,-*] ($(b.east)!.5!(xorone.in 2)$) coordinate (branch) -- (xorone.in 2)
        (c.west) to (and.out)
        (s.west) to (xorone.out)
        (branch) |- (and.in 2);  
      \end{circuitikz}
    \end{figure}
  \end{minipage}\hfill
  \begin{minipage}[c]{0.2\textwidth}
    \vspace{1.5em}
    \begin{center}
      \begin{tabular}{|c|c|c|c|}
        \hline
        $A$ & $B$ & $S$ & $C$
        \\ \hline \hline
        0 & 0 & 0 & 0
        \\ \hline
        0 & 1 & 1 & 0
        \\ \hline
        1 & 0 & 1 & 0
        \\ \hline
        1 & 1 & 0 & 1
        \\ \hline
      \end{tabular}
    \end{center}
  \end{minipage}\hfill
  \begin{minipage}[c]{0.5\textwidth}
    \begin{itemize}
      \vspace{1.5em}
      \item egy \kkod{XOR} és \kkod{AND} kapuval megvalósítható
      
      \item feladata 2 bit összeadása
      
      \item $S = \overline{A}B + A\overline{B}$
      \tabto{3.2cm} – \tabto{4cm} összeg
      
      \item $C = AB$
      \tabto{3.2cm} – \tabto{4cm} maradék (carry)
    \end{itemize}
  \end{minipage}\hfill

  \vspace{2em}
  {\large \zkod{Teljes összeadó}:}
  


  \fogalom{Szekvenciális logikai hálózatok}
  lorem
  \fogalom{Flip-flop}
  lorem


  % ----------------------------------------------------------------------------
  % ------------------------------ Section 12 ----------------------------------
  % ----------------------------------------------------------------------------
  \section{Tétel – A számítógép architektúrák alapjai} %12

  \fogalom{A számítógép felépítése}
  lorem
  \fogalom{Memóriák}
  lorem
  \fogalom{CPU részei}
  lorem
  \fogalom{Utasítás ciklus}
  lorem
  \fogalom{Szubrutinhívás}
  lorem
  \fogalom{Interrupt}
  lorem
  \fogalom{Közvetlen memória hozzáférés}
  lorem




\end{document}